%\section{Theoretical properties of the MMD flow}\label{sec:theory}



\subsection{Lambda displacement convexity of the MMD}

One important criterion to characterize the convergence of the gradient flow of a functional $\F$ is the notion of \textit{displacement convexity} of such a functional. Displacement convexity (see \cite{Villani:2004}, Definition 1). states that the functional evaluated at any distribution in a geodesic path between two distributions $\nu$ and $\nu'$ will be upper-bounded by a convex mixture of $\F(\nu)$ and $\F(\nu')$, as explained formally in the following definition.
\begin{definition}\label{def:displacement_convexity}
 Let $\mu$
and $\nu$ be two probabilities densities. There exists a $\mu-a.e.$
unique gradient of a convex function, denoted by $\nabla\phi$, such that $\nu$
is equal to $\nabla\phi_{\#}\mu$ and one can define \aknote{the displacement geodesic?} $\rho_{t}=((1-t)Id+t\nabla\phi)_{\#}\mu$
for $0\leq t\leq1$. We say that a functional $\nu\mapsto\mathcal{F}(\nu)$
is displacement convex if 
\[
t\mapsto\mathcal{F}(\rho_{t})
\]
 is convex for any $\mu$ and $\nu$. Moreover, we say that $\mathcal{F}$
is displacement convex in a neighborhood of $\mu$ if there exists a radius $r>0$
such that the above property holds for any $\nu$ with $W_{2}(\mu,\nu)\leq r$.
\end{definition}


This notion of convexity is to be related to the more widely used notion of convexity called \textit{mixture convexity}:
\begin{align}
	\F(t\nu +(1-t)\nu')\leq t\F(\nu)+(1-t)\F(\nu') \qquad t\in [0,1]
\end{align}
%Unlike mixture convexity, displacement convexity is compatible with the $W_2$ metric and is therefore the natural notion to use for characterizing convergence of gradient flows in the $W_2$ metric.
Although mixture convexity holds for $\F$ (see \cref{lem:mixture_convexity}), this property is less critical for characterizing convergence of gradient flows in the $W_2$ metric. On the other hand, displacement convexity is compatible with the $W_2$ metric \cite{Bottou:2017} and is therefore the natural notion to use in our setting. Unfortunately, $\F$ fails to be displacement convex in general. Instead we will show that $\F$ satisfies some weaker notion of convexity called $\Lambda$-displacement convexity:
%
\begin{definition}\label{def:lambda-convexity}
($\Lambda$-convexity \cite{Villani:2009} Definition 16.4). Let $(\mu,v)\mapsto\Lambda(\mu,v)$
be a function that defines for each probability distribution $\mu$
a quadratic form on the set of square integrable vectors valued functions
$v$ , i.e: $v\in L_{2}(\mathbb{R}^{d},\mathbb{R}^{d},\mu)$ . We
further assume that:
\[
\inf_{\mu,v}\frac{\Lambda(\mu,v)}{\Vert v\Vert_{L_{2}(\mu)}^{2}}>-\infty.
\]
We say that a functional $\mu\mapsto\mathcal{F}(\mu)$ is $\Lambda$-convex
if for any $\mu$ and $\nu$ and a minimizing geodesic $\text{\ensuremath{\rho_{t}}}$
between $\mu$ and $\nu$ with velocity vector field $v_{t}$, i.e:
$\partial_{t}\rho_{t}+div(\rho_{t}v_{t})=0;\rho_{0}=\mu;\rho_{1}=\nu$
the following holds:
\begin{equation*}
\frac{d^{2}\mathcal{F}(\rho_{t})}{dt^{2}}\geq\Lambda(\rho_{t},v_{t})\qquad\forall t\in[0,1].
\end{equation*}
\end{definition}

To show the $\Lambda$-convexity of the functional defined in \cref{eq:MMD_functional} we first make the following assumptions on the kernel:
\begin{assumplist} 
\item \label{assump:bounded_trace} $ \vert \sum_{1\leq i\leq d} \partial_i\partial_ik(x,x) \vert\leq \frac{L}{3}  $ for all $x\in \mathbb{R}^d$.
\item \label{assump:bounded_hessian} $\Vert H_xk(x,y) \Vert_{op} \leq \frac{L}{3}$ for all $x,y\in \mathbb{R}^d$, where $H_xk(x,y)$ is the hessian of $x\mapsto k(x,y)$ and $\Vert.\Vert_{op}$ is the operator norm.
\item \label{assump:bounded_fourth_oder} $\Vert Dk(x,y) \Vert\leq \lambda  $ for all $x,y\in \mathbb{R}$, where $Dk(x,y)$ is an $\mathbb{R}^{d^2}\times \mathbb{R}^{d^2}$ matrix with entries given by $\partial_{x_{i}}\partial_{x_{j}}\partial_{x'_{i}}\partial_{x_{j}'}k(x,x')$.
\end{assumplist}\aknote{do we have an order of magnitude for lambda? or just we put a remark to say it's satisfied by the gaussian kernel}
The next proposition states that the functional defined in \cref{eq:MMD_functional} is $\Lambda$-displacement convex and provide and explicit expression for the functional $\Lambda$.

\begin{proposition}
\label{prop:lambda_convexity} Suppose \cref{assump:bounded_fourth_oder} is satisfied for some $\lambda \in \R^+$. The functional $\nu\mapsto \F(\nu)$ is $\text{\ensuremath{\Lambda}}$-convex
with $\Lambda$ given by:
\begin{equation}
\Lambda(\rho,v)=\langle v,(C_{\rho}-\lambda \F(\rho)^{\frac{1}{2}}I)v\rangle_{L_{2}(\rho)}\label{eq:Lambda}
\end{equation}
where $C_{\rho}$ is the (positive) operator defined by:
\begin{align}\label{eq:positive_operator_C}
	(C_{\rho}v)(x)=\int\nabla_{x}\nabla_{x'}k(x,x')v(x')d\rho(x')
\end{align}
\end{proposition}
%
%
Consider the geodesic \aknote{path/geodesic/curve?}$\rho_{t}=((1-t)Id+t\nabla\phi)_{\#}\mu$ of \cref{def:displacement_convexity}. It is worth noting that $\rho_{0}=\mu$ and at time $t=0$ we have
that $\F(\rho_{0})=0$, hence we get:
\[
\frac{d^{2}\F(\rho_{t})}{dt^{2}}\vert_{t=0}=\langle v_{t},C_{\rho_{t}}v_{t}\rangle_{L_{2}(\rho_{t})}\geq0.
\]
This shows that $\nu\mapsto \F(\nu)$ has a non-negative
hessian at $\mu$ which is not surprising since $\mu$ is the global
minimum of this functional.
\begin{corollary}\label{cor:integral_lambda_convexity}
For any geodesic $\rho_{t}$ between two probability distributions
$\rho_{0}$ and $\rho_{1}$ the following holds:
\begin{equation}
\F(\rho_{t})\leq(1-t)\F(\rho_{0})+t\F(\rho_{1})-\int_{0}^{1}\Lambda(\rho_{s},v_{s})G(s,t)ds\label{eq:integral_lambda_convexity}
\end{equation}
where $\Lambda$ is given by \cref{eq:Lambda} and $G$ is given
by:
\[
G(s,t)=\begin{cases}
s(1-t) & s\leq t\\
t(1-s) & s\geq t
\end{cases}
\]
\end{corollary}
%

\begin{corollary}
\label{cor:loser_bound}Assume the distributions are supported on
$\mathcal{X}$ and the kernel is bounded, i.e: $\sup_{x,y\in\mathcal{X}}\vert k(x,y)\vert<\infty$.
Then the following holds:
\begin{equation}
\F(\rho_{t})\leq(1-t)\F(\rho_{0})+t\F(\rho_{1})+t(1-t)K
\end{equation}
where $K$ is a constant depending on $\X$ and the kernel $k$ in $\F$.
\end{corollary}
%
%
\cref{cor:loser_bound}, is a loser bound and does not account for the local
convexity of the MMD. However, it allows to state the following result,
which is inspired from (\cite{Bottou:2017}, Theorem 6.3) but generalizes
it to the case of 'almost convex' functionals.
\begin{proposition}
\label{prop:almost_convex_optimization}
(Almost convex optimization). Let $\mathcal{P}$ be a closed subset
of $\mathcal{P}(\mathcal{X})$ which is displacement convex\aknote{weird for a set to be displacement convex? it was for functionals}. Then
for all $M>\inf_{\rho\in\mathcal{P}}\F(\rho)+K$, the following
holds:
\end{proposition}
\begin{enumerate}
\item The level set $L(\mathcal{P},M)=\{\rho\in\mathcal{P}:\F(\rho)\leq M\}$
is connected
\item For all $\rho_{0}\in\mathcal{P}$ such that $\F(\rho_0)>M$
and all $\epsilon>0$, there exists $\rho\in\mathcal{P}$ such that
$W_{2}(\rho,\rho_{0})=\mathcal{O}(\epsilon)$ and
\[
\F(\rho)\leq \F(\rho_{0})-\epsilon(\F(\rho_{0})-M).
\]
\end{enumerate}
%
\begin{remark}
The result in \Cref{prop:almost_convex_optimization} means that it is possible to optimize the cost function $\rho\mapsto \F(\rho)$
on $\mathcal{P}$ as long as the barrier $\inf_{\rho\in\mathcal{P}}\F(\rho)+K$
is not reached. We provide now a simple proof of this result.
\end{remark}


\begin{remark}
	A possible direction would be to directly leverage the tighter inequality in \cref{eq:integral_lambda_convexity} to get a better description of the loss landscape.
\end{remark}





