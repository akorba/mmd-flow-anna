
\subsection{Lojasiewicz type inequality - Convergence of the continuous flow}

Here we would like to derive an inequality between the time derivative of the Lyapounov functional $\mathcal{F}$ along its gradient flow $t\mapsto \nu_t$. For this purpose we first introduce the weighted negative Sobolev distance \manote{cite villani and peyre and Mroueh}:
\begin{align}\label{eq:neg_sobolev}
	\Vert \nu - \mu \Vert_{\dot{H}^{-1}(\nu)} = \sup_{\substack{ f\in W_0^{1,2}(\nu) \\ \nu(\Vert \nabla f \Vert^2) \leq 1 }} \vert \nu(f)-\mu(f)\vert 
\end{align}
Where $W_0^{1,2}(\nu)$ is the space $1$ order Sobolev functions with functions vanishing at the boundary of the domain.
The distance defined in \cref{eq:neg_sobolev} plays a fundamental role in dynamic optimal transport as it linearizes the $W_2$ distance when $\mu$ is arbitrarily close to $\nu$. It can also be seen as the minimum kinetic energy needed to advect the mass $\nu$ to $\mu$. However, this quantity might be infinite \manote{say exactly when it is finite} and one of the key problems would be to control its value during the evolution of the flow. More precisely we will rely on the following statement:
\begin{align}\label{eq:bounded_neg_sobolev}
	\Vert \nu_t  - \mu \Vert_{\dot{H}^{-1}(\nu_t)} \leq C \qquad \forall t\geq 0.
\end{align} 
where $\nu_t$ is defined by the gradient flow and $\mu$ is the target distribution. When \cref{eq:bounded_neg_sobolev}  holds, we have the following proposition:
\begin{proposition}\label{prop:PL_type_inequality}
	When \cref{eq:bounded_neg_sobolev} holds, the following inequality is then satisfied at all times:
	\begin{align}\label{eq:PL_type_inequality}
		\Vert \nabla f_t \Vert_{L_2(\nu_t)} \geq \frac{1}{C} \Vert f_t \Vert^2_{\mathcal{H}} \qquad \forall t\geq 0.
	\end{align}
\end{proposition}
\begin{proof}
	Indeed, this follows simply from the definition of the negative Sobolev distance: Consider $g = \Vert \nabla f_t\Vert^{-1}_{L_2(\nu_t)} f_t$, then $g\in W_0^{1,2}(\nu)$ \manote{this suggests an assumption on the kernel so that all those function satisfy a boundary condition} and $\Vert \nabla g \Vert_{L_2(\nu_t)}\leq 1$. Therefore, we directly have:
	\begin{align}
		\Vert \nu_t - \mu\Vert_{\dot{H}^{-1}(\nu_t)}\geq \vert \nu_t(g) - \mu(g)  \vert.
	\end{align}
Now, recall the definition of $g$, which implies that
\[
\vert \nu_t(g) - \mu(g)  \vert = \Vert \nabla f_t\Vert^{-1}_{L_2(\nu_t)} \vert \nu_t(f_t)-\mu(f_t)\vert.
\]
But since $f_t$  is exactly the witness functions between $\nu_t$ and $\mu$, it follows that $\nu_t(f_t)-\mu(f_t) = \Vert f_t\Vert^2_{\kH}$.
Using \cref{eq:bounded_neg_sobolev}, we get the desired inequality.
\end{proof}

Now we will use the inequality in \cref{prop:PL_type_inequality} to prove a convergence result towards the global optimum $\mu$. This is provided in \cref{prop:convergence}.

\begin{proposition}\label{prop:convergence}
	If \cref{eq:bounded_neg_sobolev} is satisfied for all times then $t\mapsto \mathcal{F}(\nu_t)$ converges to $0$ with a rate of convergence given by:
	\begin{align}
		\mathcal{F}(\nu_t)\leq \frac{1}{\mathcal{F}(\nu_0)^{-1} + \frac{4t}{C}}
	\end{align}
\end{proposition}
\begin{proof}
	The proof is a simple consequence of \cref{prop:mmd_flow,eq:bounded_neg_sobolev}. Indeed, by \cref{prop:mmd_flow} we have that 
	\begin{align}
		\dot{\F}(\nu_t) = - \Vert \nabla f_t \Vert^2_{L_2(\nu_t)} 	
	\end{align}
	Using \cref{eq:PL_type_inequality}, we directly get that:
	\begin{align}
		\dot{\F}(\nu_t) \leq  -\frac{4}{C}\F(\nu_t)^2
	\end{align}
It is clear that if $\mathcal{F}(\nu_0)>0$ then $\F(\nu_t)>0$ at all times by uniqueness of the solution. Hence, one can divide by $\F(\nu_t)^2$ and integrate the inequality from $0$ to some time $t$. The desired inequality is obtained by simple calculations.
\end{proof}

All the difficulty is to see now when \cref{eq:bounded_neg_sobolev} holds. One possible strategy would be to start from initial $\nu_0$ such that $\Vert \nu_0  - \mu \Vert_{\dot{H}^{-1}(\nu_0)} \leq C $  for some finite positive value $C$ and then show that this property is preserved during the dynamics. It is also possible to have a time depended constant $C_t$ as long as its growth is such that:
\begin{align}
	\lim_{t\rightarrow +\infty} \int_0^t C_s^{-1}\diff s = +\infty
\end{align}
For instance $C_t$ could have up to a linear growth in time. In this case the decay of $\F(\nu_t)$ will no longer be in $\frac{1}{t}$ but only in $\frac{1}{\log(t)}$ \manote{This seems unlikely if we end up having convergence of $\nu_t$, but who nows.}.
One possible promising condition for \cref{eq:bounded_neg_sobolev} to hold would be if $\mu \ll \nu_0$ and if this property is preserved during the dynamics.


 








