
\subsection{Additional mathematical background}

\subsubsection{Stochastic processes}\label{sec:ito_stochastic}

Consider the Itô process, i.e. the stochastic process:
\begin{equation}
dX_t=g(X_t)dt.
\end{equation}
Let $f$ be a twice-differentiable scalar function, Itô's formula (see \cite{ito1951stochastic}) can be written:
\begin{equation}
df(X_t)=\nabla f(X_t).g(X_t)dt
\end{equation}
Let $\rho_t$ be the distribution of the process $X_t$. We have:
\begin{align}
\E[\frac{df}{dt}(X_t)]&= \E[\nabla f(X_t).g(X_t)]\\
\Longleftrightarrow \int f(X) \frac{d \rho_t}{dt}(X)&=-\int f(X)div(g(X)\rho_t(X))
\end{align}
where the second line is obtained by integrating by parts on both sides of the equality. Finally, the distribution $\rho_t$ verifies the continuity equation: 
\begin{equation}
\frac{d\rho_t}{dt}=div(g\rho_t)
\end{equation}


\subsubsection{Additional lemmas}

\begin{lemma}\label{lem:mixture_convexity}
	The functional $\F$ is mixture convex: for any probability distributions $\nu_1$ and $\nu_2$ and scalar $1\leq \lambda\leq 1$:
	\begin{align*}
	\F(\lambda \nu_1+(1-\lambda)\nu_2)\leq \lambda \F(\nu_1)+ (1-\lambda)\F(\nu_2)
	\end{align*}
\end{lemma}
\begin{proof}
	Let $\nu$ and $\nu'$ be two probability distributions and $0\leq \lambda\leq 1$.
	We need to show that \[\mathcal{F}(\lambda \nu + (1-\lambda)\nu') -\lambda \mathcal{F}(\nu) -(1-\lambda)\mathcal{F}(\nu')\leq 0\]
	This follows from a simple computation which shows that:
	\begin{align*}
	\mathcal{F}(\lambda \nu + (1-\lambda)\nu') -\lambda \mathcal{F}(\nu) -(1-\lambda)\mathcal{F}(\nu') = -\frac{1}{2}\lambda(1-\lambda)MMD(\nu,\nu')^2 \leq 0.
	\end{align*}
\end{proof}


\begin{lemma}\label{lem:mmd_w2}
	 Suppose that $k$ is bounded and measurable on $\X$, and that there exists $L_k$ such that $\forall x,y \in \X$, $\| k(x,.)-k(y,.) \|_{\kH}\le L_k \|x-y\|$. Then for all $\mu, \nu$ in $\mathcal{P}(\X)$:
	\begin{equation}
	MMD^2(\mu,\nu)\le  L_k W_1^2(\mu,\nu) \le L_k W_2^2(\mu,\nu)
	\end{equation}
\end{lemma}
\begin{proof}
Let $\mu, \nu$ in $\mathcal{P}(\X)$. By Proposition 20 in \cite{sriperumbudur2010hilbert} we have:
\begin{equation}
	MMD(\mu, \nu)	 \le \inf_{\pi \in \Pi(\mu, \nu)} \int \| k(x,.)-k(y,.) \|_{\kH} d\pi(\mu, \nu)
\end{equation}
Hence:
\begin{align}
	MMD^2(\mu, \nu)	
	 \le (\inf_{\pi \in \Pi(\mu, \nu)} \int L_k \| x-y \| d\pi(\mu, \nu))^2
 \le L_k^2 W_1^2(\mu, \nu) \le L_k^2 W_2^2(\mu,\nu)
\end{align}
\end{proof}
