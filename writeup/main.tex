\documentclass{article}
\usepackage[T1]{fontenc}
\usepackage[utf8]{inputenc}
\usepackage[american]{babel}
\usepackage[nonatbib]{neurips_2019}
\usepackage{amsmath}
\usepackage{amsthm}
\usepackage{url}
\usepackage{booktabs}
\usepackage{amsfonts}
\usepackage{nicefrac}
\usepackage{microtype}
\usepackage{enumitem}
\usepackage{etoolbox}
\usepackage{amstext}
\usepackage{amssymb}
\usepackage{csquotes}
\usepackage{stmaryrd}
\usepackage{xparse}
\usepackage{bbm}
%%\usepackage{color}
\usepackage{xcolor}
%\usepackage[]{todonotes}
%\setlength{\marginparwidth}{2cm}

\usepackage{appendix}
\usepackage{graphicx}
\usepackage{epstopdf}
\usepackage{relsize}
\usepackage{subcaption}


\usepackage{algorithm}
\usepackage[noend]{algpseudocode}
\usepackage{setspace}

\makeatletter
\def\BState{\State\hskip-\ALG@thistlm}
\makeatother

\makeatletter
\newcommand*{\skipnumber}[2][1]{%
	{\renewcommand*{\alglinenumber}[1]{}\State #2}%
	\addtocounter{ALG@line}{-#1}}
\makeatother
\newcommand\mysim{\mathrel{\stackrel{\makebox[0pt]{\mbox{\normalfont\tiny i.i.d}}}{\sim}}}


\usepackage[textwidth=2cm, textsize=footnotesize]{todonotes}  
%\setlength{\marginparwidth}{1.5cm}               %  this goes with todonotes

%\usepackage{amsthm}
%\usepackage[noabbrev,capitalize]{cleveref}

\usepackage[colorlinks,linkcolor={red!80!black},citecolor={green!80!black},urlcolor={blue!80!black},hypertexnames=false]{hyperref}

\usepackage[noabbrev,capitalize]{cleveref}
%% make \cref{whatever} render like \eqref{whatever}
\crefformat{equation}{(#2#1#3)}
\crefrangeformat{equation}{(#3#1#4) to~(#5#2#6)}
\crefmultiformat{equation}{(#2#1#3)}{ and~(#2#1#3)}{, (#2#1#3)}{ and~(#2#1#3)}
%% https://tex.stackexchange.com/a/121055/9019
\crefname{appsec}{Appendix}{Appendices}

% \usepackage{autonum}

%\renewcommand\equationautorefname{\@gobble}
%\usepackage[backend=biber,style=numeric,maxcitenames=2,maxbibnames=8,sortcites,isbn=false,doi=false,giveninits=true]{biblatex}
%\renewbibmacro{in:}{}
%\usepackage{utils/patch_biblatex}
\usepackage[backend=bibtex,style=numeric,maxcitenames=2,maxbibnames=8,sortcites,isbn=false,doi=false,giveninits=true]{biblatex}
%\usepackage[backend=bibtex]{biblatex}


\addbibresource{biblio.bib}


% Example definitions.
% --------------------
\def\x{{\mathbf x}}
\def\L{{\cal L}}

\DeclareMathOperator{\var}{\mathbb Var}
\DeclareMathOperator{\cov}{cov}
\DeclareMathOperator{\rank}{rank}
\DeclareMathOperator{\dom}{dom}
\DeclareMathOperator{\zer}{zer}
\DeclareMathOperator{\aver}{av}
\DeclareMathOperator{\inter}{int}
\DeclareMathOperator{\relint}{ri}
\DeclareMathOperator{\epi}{epi}
\DeclareMathOperator{\graph}{gr}
\DeclareMathOperator{\prox}{prox}
\DeclareMathOperator{\tr}{tr}
\DeclareMathOperator{\support}{supp}
\DeclareMathOperator{\dist}{dist}
\DeclareMathOperator{\lev}{lev}
\DeclareMathOperator{\rec}{rec}
\DeclareMathOperator{\cl}{cl}
\DeclareMathOperator{\co}{co}
\DeclareMathOperator{\clo}{\overline co}
\DeclareMathOperator{\distC}{\mathsf d}
\DeclareMathOperator*{\diag}{diag}
\newcommand{\KL}{\mathop{\mathrm{KL}}\nolimits}


\newcommand{\leftnorm}{\left|\!\left|\!\left|}
\newcommand{\rightnorm}{\right|\!\right|\!\right|}

%\newcommand{\eqdef}{{\stackrel{\text{def}}{=}}} 
\newcommand{\eqdef}{:=} 

\newcommand{\1}{\mathbbm 1}
\newcommand{\bs}{\boldsymbol}

\newcommand{\itpx}{{\mathsf x}}
\newcommand{\sx}{{\mathsf x}}
\newcommand{\sy}{{\mathsf y}}
\newcommand{\sz}{{\mathsf z}}
\newcommand{\sw}{{\mathsf w}}
\newcommand{\sF}{{\mathsf F}}
\newcommand{\sH}{{\mathsf H}}

\newcommand{\ZZ}{\mathbb Z}
\newcommand{\CC}{\mathbb{C}}
\newcommand{\bP}{{{\mathbb P}}} 
\newcommand{\bE}{{{\mathbb E}}} 
\newcommand{\bV}{{{\mathbb V}}} 
\newcommand{\bN}{{{\mathbb N}}} 

% Operators, domains, etc.  
\newcommand{\mA}{{\mathcal A}} 
\newcommand{\mB}{{\mathcal B}} 
\newcommand{\mC}{{\mathcal C}} 
\newcommand{\mD}{{\mathcal D}} 
\newcommand{\mO}{{\mathcal O}} 
\newcommand{\mU}{{\mathcal U}}
\newcommand{\mX}{{\mathcal X}}
\newcommand{\mY}{{\mathcal Y}}
\newcommand{\mZ}{{\mathcal Z}} 
\newcommand{\bmD}{\cl({\mathcal D})} 

\newcommand{\sA}{{\mathsf A}}
\newcommand{\sB}{{\mathsf B}}
\newcommand{\sJ}{{\mathsf J}}
\newcommand{\sX}{{\mathsf X}}
\newcommand{\sG}{{\mathsf G}}
\newcommand{\sY}{{\mathsf Y}}

\newcommand{\maxmon}{{\mathscr M}} 
\newcommand{\Selec}{{\mathfrak S}} 

% Sigma fields
\newcommand{\mcA}{{\mathscr A}} 
\newcommand{\mcB}{{\mathscr B}} 
\newcommand{\mcN}{{\mathscr N}} 
\newcommand{\mcT}{{\mathscr T}} 
\newcommand{\mcI}{{\mathscr I}} 
\newcommand{\mcF}{{\mathscr F}} 
\newcommand{\mcG}{{\mathscr G}} 
\newcommand{\mcX}{{\mathscr X}} 
\newcommand{\cP}{{{\mathcal P}}} 
\newcommand{\cS}{{{\mathcal S}}} 
\newcommand{\cZ}{{{\mathcal Z}}} 
\newcommand{\cF}{{{\mathcal F}}} 
\newcommand{\cG}{{{\mathcal G}}} 
\newcommand{\cM}{{{\mathcal M}}} 
\newcommand{\cD}{{{\mathcal D}}} 
\newcommand{\cE}{{{\mathcal E}}} 
\newcommand{\cL}{{{\mathcal L}}}
\newcommand{\cT}{{{\mathcal T}}} 
\newcommand{\cN}{{{\mathcal N}}} 
\newcommand{\cK}{{{\mathcal K}}} 
\newcommand{\cI}{{{\mathcal I}}} 

% Spaces 
\newcommand{\R}{{{\mathbb R}}} 
\newcommand{\E}{{{\mathbb E}}} 
\newcommand{\kH}{{{\mathcal H}}} 
\newcommand{\X}{{{\mathcal X}}} 
\newcommand{\F}{{{\mathcal F}}} 
\newcommand{\Hil}{E}                % Hilbert   
\newcommand{\Ban}{E}                % Banach   
%\newcommand{\RN}{{{\mathbb R}^N}} 
\newcommand{\bR}{{{\mathbb R}}} 

\newcommand{\m}{\mathfrak{m}}
\newcommand{\toL}{\xrightarrow[]{{\mathcal L}}}
\newcommand{\toweak}{\xrightharpoonup[]{{\mathcal L}}}

\newcommand{\ps}[1]{\langle #1 \rangle}
\newcommand{\psh}[1]{\langle #1 \rangle_{\kH}}
% 
% Almost sure convergence
\newcommand{\toasshort}{\stackrel{\text{as}}{\to}}
\newcommand{\toaslong}{\xrightarrow[n\to\infty]{\text{a.s.}}}

% Convergence in probability 
\newcommand{\toprobashort}{\,\stackrel{\mathcal{P}}{\to}\,}
\newcommand{\toprobalong}{\xrightarrow[n\to\infty]{\mathcal P}}
%
% Convergence in law 
\newcommand{\todistshort}{{\stackrel{\mathcal{D}}{\to}}}
\newcommand{\todistlong}{\xrightarrow[n\to\infty]{\mathcal D}}


\newcommand{\aknote}[1]{\todo[color=cyan!20]{#1}}
\newcommand{\asnote}[1]{\todo[color=green!20]{#1}}
\newcommand{\manote}[1]{\todo[color=magenta]{#1}}
\newcommand{\agnote}[1]{\todo[color=red]{#1}}

\newcommand*\diff{\mathop{}\!\mathrm{d}}
\newcommand*\Diff[1]{\mathop{}\!\mathrm{d^#1}}

%Moreau
\newcommand{\my}{{{\nabla ^\gamma g}}}
\newcommand{\myn}{{{\nabla ^{\gamma_{n+1}} g}}}
\def\macom#1{{\textcolor{red}{[MA: #1]}}}
%% ==============================================================


%\theoremstyle{definition}
\makeatother
\newtheorem{theorem}{Theorem}
\newtheorem{lemma}[theorem]{Lemma}
\newtheorem{corollary}[theorem]{Corollary}
\newtheorem{proposition}[theorem]{Proposition}
\newtheorem{definition}{Definition}
\newtheorem{remark}{Remark}
\newtheorem{condition}{Condition}
\newtheorem{assumption}{Assumption}
\newtheorem{example}{Example}

%% make \cref{whatever} render like \eqref{whatever}
\crefformat{equation}{(#2#1#3)}
\crefrangeformat{equation}{(#3#1#4) to~(#5#2#6)}
\crefmultiformat{equation}{(#2#1#3)}{ and~(#2#1#3)}{, (#2#1#3)}{ and~(#2#1#3)}
%% https://tex.stackexchange.com/a/121055/9019


\newcommand\numberthis{\addtocounter{equation}{1}\tag{\theequation}}

\newlist{assumplist}{enumerate}{1}
\setlist[assumplist]{label=(\textbf{\Alph*})}
\Crefname{assumplisti}{Assumption}{Assumptions}

\newlist{assumplist2}{enumerate}{1}
\setlist[assumplist2]{label=(\textbf{\Roman*})}
\Crefname{assumplist2i}{Assumption}{Assumptions}

\newlist{proplist}{enumerate}{1}
\setlist[proplist]{label=({\roman*})}
\Crefname{proplisti}{Property}{Properties}



\title{Maximum Mean Discrepancy Gradient Flow}

\begin{document}
\maketitle


\iffalse
%AG: old abstract

\begin{abstract}
The Maximum Mean Discrepancy (MMD) was successfully used as a loss functional to train generative models. In a non-parametric setting, the MMD can also be used as a loss function to learn distributions using optimal transport theory.
In this work, we construct a Wasserstein gradient flow of the MMD and provide an algorithm to simulate such flow. The proposed algorithm is based on a space-time discretization of the theoretical gradient flow of the MMD and aims at finding the best probability distribution that is close to the data as much as possible. We analyze the convergence properties of the gradient flow towards a global optimum and provide a simple algorithmic fix to improve convergence. We also show that the discretized algorithm approaches the gradient flow of the MMD as the sample size increases.
\end{abstract}


\begin{abstract}
  In this work, we construct a Wasserstein gradient flow of the maximum mean discrepancy (MMD), and provide an algorithm to simulate such flow.
The MMD is an integral probability metric defined for a reproducing kernel Hilbert space (RKHS), and serves as a metric on probability measures for a sufficiently rich RKHS. Notably, both the MMD and its gradient have simple closed-form expressions.
  The proposed algorithm is based on a space-time discretization of the theoretical gradient flow of the MMD, and aims at finding a  distribution of particles that is as  close as possible to a target distribution, where the target may itself be represented by a sample. We analyze the convergence properties of the gradient flow towards a global optimum and provide a simple algorithmic fix to improve convergence. We also show that the discretized algorithm approaches the gradient flow of the MMD as the sample size increases.
\end{abstract}


% Abstract submission abstract:
\begin{abstract}
	In this work, we construct a Wasserstein gradient flow of the maximum mean discrepancy (MMD), and provide an algorithm to simulate this flow.
	The MMD is an integral probability metric defined for a reproducing kernel Hilbert space (RKHS), and serves as a metric on probability measures for a sufficiently rich RKHS.
	The proposed algorithm is based on a space-time discretization of the theoretical gradient flow of the MMD, and aims at finding a  distribution of particles that is as  close as possible to a target distribution, where the target may itself be represented by a sample.
	The practical implementation of the flow is straightforward, since both the MMD and its gradient have simple closed-form expressions.
	We analyze the convergence properties of the gradient flow towards a global optimum, and  show that the discretized algorithm approaches the gradient flow of the MMD as the sample size increases. Finally, we provide a simple algorithmic fix to ensure convergence under more general conditions.
\end{abstract}

\fi

\iffalse
\begin{abstract}
  In this work, we construct a Wasserstein gradient flow of the maximum mean discrepancy (MMD) and study its convergence properties.
  The MMD is an integral probability metric defined for a reproducing kernel Hilbert space (RKHS), and serves as a metric on probability measures for a sufficiently rich RKHS. The practical implementation of the flow is straightforward, since both the MMD and its gradient have simple closed-form expressions.Moreover, it can be easily estimated with samples. We exhibit conditions for convergence of the gradient flow towards a global optimum, that can be related to particle transport when optimizing neural networks.
  %The proposed algorithm is based on a space-time discretization of the theoretical gradient flow of the MMD, and aims at finding a  distribution of particles that is as  close as possible to a target distribution, where the target may itself be represented by a sample.
  We also propose a way to regularize the MMD flow, based on an injection of noise in the gradient. This simple algorithmic fix comes with theoretical and empirical evidence.
\end{abstract}

\fi

\begin{abstract}
  We construct a Wasserstein gradient flow of the maximum mean discrepancy (MMD) and study its convergence properties.
  The MMD is an integral probability metric defined for a reproducing kernel Hilbert space (RKHS), and serves as a metric on probability measures for a sufficiently rich RKHS.  We obtain conditions for convergence of the gradient flow towards a global optimum, that can be related to particle transport when optimizing neural networks.
  %The proposed algorithm is based on a space-time discretization of the theoretical gradient flow of the MMD, and aims at finding a  distribution of particles that is as  close as possible to a target distribution, where the target may itself be represented by a sample.
  We also propose a way to regularize this MMD flow, based on an injection of noise in the gradient. This algorithmic fix comes with theoretical and empirical evidence.
The practical implementation of the flow is straightforward, since both the MMD and its gradient have simple closed-form expressions which can be estimated with samples.
  
  
\end{abstract}


\section{Introduction}

This paper deals with the problem of sampling from a probability measure $\mu$ on $(\R^d,\mathcal{B}(\R^d))$ which admits a density, still denoted by $\mu$, with respect to the Lebesgue measure.
This problem appears in machine learning, Bayesian inference, computational physics... Classical methods to tackle this issue are Markov Chain Monte Carlo methods, for instance Metropolis-Hastings algorithm, Gibbs sampling. The main drawback of these methods is that one needs to choose an appropriate proposal distribution, which is not trivial. Consequently, other algorithms based on continuous dynamics have been proposed, such as the over-damped Langevin diffusion:
\begin{equation}\label{eq:langevin_diffusion}
dX_t= -\nabla \log \mu (X_t)dt+\sqrt{2}dB_t
\end{equation}
where $(B_t)_{t\ge0}$ is a $d$-dimensional Brownian motion. The Langevin Monte-Carlo (LMC) algorithm, or Unadjusted Langevin algorithm (ULA) considers the Markov chain $(X_k)_{k\ge1 }$ given by the Euler-Maruyama discretization of the diffusion \eqref{eq:langevin_diffusion}:
\begin{equation}
X_{k+1} = X_k - \gamma_{k+1}\nabla \log \mu(X_k) + \sqrt{2\gamma_{k+1}G_{k+1}}
\end{equation}
where $(\gamma_k)_{k\ge1}$ is a sequence of step sizes (constant or convergent to zero), and
$(G_k)_{k \ge 1}$ is a sequence of i.i.d. standard $d$-dimensional Gaussian random variables. This algorithm has attracted a lot of attention... But....\aknote{say something about the requirement of the knowledge of gradient of log target and how it is difficult to estimate?}

Interestingly, it has been shown in \cite{jordan1998variational} that the family of distributions $(\rho_t)_{t\ge 0}$ where $\rho_t$ is the distribution of the process \eqref{eq:langevin_diffusion} is the solution of a gradient
flow equation in the Wasserstein space of order 2 associated with a particular functional, the KL-divergence. Langevin Monte-Carlo can thus be formulated as a first order optimization algorithm of the KL-divergence defined on the Wasserstein space of order 2 (see also \cite{durmus2018analysis,bernton2018langevin}). Inspired by this interpretation, we propose a new method to sample from a distribution, discretizing a gradient
flow equation in the Wasserstein space of order 2 associated with another well-chosen functional, the Maximum Mean Discrepancy, introduced in \cite{gretton2012kernel}. To the best of our knowledge, this work is the first one which investigates the flow of a discrepancy between distributions different from the Kullback-Leibler divergence. \aknote{true? (also, reformulate this sentence)}


This paper is organized as follows. In \cref{sec:preliminaries}, definitions and mathematical background needed in the paper are introduced, and \cref{sec:mmd_flow} is devoted to deriving the MMD flow and the associated sampling algorithm.
\cref{sec:theory} investigates at length the theoretical properties of this flow. 


\section{Preliminaries}\label{sec:preliminaries}

In this section we recall how to endow the space of probability measures $\mathcal{P}(\X)$ on $\X$ a convex  subset of $\R^d$ with a distance (e.g, optimal transport distances), and then deal with gradient flows of suitable functionals on such a metric space. The reader may refer to the recent review of \cite{santambrogio2017euclidean} for further details. For a given distributions $\nu\in\mathcal{P}_2(\X)$ and an integrable function $f$ under $\nu$, the expectation of $f$ under $\nu$ will be written either as $\nu(f)$ or $\int f \diff\nu$ depending on the context. 

\subsection{Background on optimal transport}

\subsubsection{$2$-Wasserstein geometry}\label{subsec:wasserstein_flow}

Let $\X \subset \R^d$. Let $\mathcal{P}_2(\X)$ the set of probability distributions on $\X$ with finite second moment.
Let $T: \X \rightarrow \X$ be a measurable map, and $\rho \in \mathcal{P}(\X)$. The push-forward measure $T_{\#}\rho$
is characterized by:
\begin{align*}
	&\quad T_{\#}\rho(A) = \rho(T^{-1}(A)) \text{ for every measurable set A,}\\
\text{or}& \int_{y \in \X} \phi(y) d(T_{\#}\rho)(y) =\int_{x \in \X}\phi(T(x)) d\rho(x) \text{ for every measurable function $\phi$.}
\end{align*}
For two given probability distributions $\nu$ and $\mu$ in $\mathcal{P}_2(\X)$ we denote by $\Pi(\nu,\mu)$ the set of possible couplings between $\nu$ and $\mu$. In other words $\Pi(\nu,\mu)$ contains all possible distributions $\pi$ on $\X\times \X$ such that if $(X,Y) \sim \pi $ then $X \sim \nu $ and $Y\sim \mu$. The $2$-Wasserstein distance on $\mathcal{P}_2(\X)$ is defined by means of optimal coupling between $\nu$ and $\mu$ in the following way:
\begin{align}\label{eq:wasserstein_2}
W_2^2(\nu,\mu) := \inf_{\pi\in\Pi(\nu,\mu)} \int \Vert x - y\Vert^2 d\pi(x,y) \qquad \forall \nu, \mu\in \mathcal{P}_2(\X)
\end{align}
It is a well established fact that such optimal coupling $\pi^*$ exists. Moreover, it can be used to define a path $(\rho_t)_{t\in [0,1]}$ between $\nu$ and $\mu$ in $\mathcal{P}_2(\X)$. For a given time $t$ in $[0,1]$ and given a sample $(x,y)$ from $\pi^{*}$, it possible to construct a sample $z_t$ from $\rho_t$ by taking the convex combination of $x$ and $y$: $z_t = s_t(x,y)$ where $s_t$ is given by \cref{eq:convex_combination}
\begin{equation}\label{eq:convex_combination}
s_t(x,y) = (1-t)x+ty \qquad \forall x,y\in \X, \; \forall t\in [0,1].
\end{equation}
The function $s_t$ is well defined since $\X$ is a convex set. More formally, $\rho_t$ can be written as the projection or push-forward of the optimal coupling $\pi^{*}$ by $s_t$:    \aknote{weird. st not operator}
\begin{equation}\label{eq:displacement_geodesic}
\rho_t = (s_t)_{\#}\pi^{*}
\end{equation}
It is easy to see that \cref{eq:displacement_geodesic} satisfies the following boundary conditions:
\begin{align}\label{eq:boundary_conditions}
\rho_0 = \nu \qquad \rho_1 = \mu.
\end{align}
Paths of the form of \cref{eq:displacement_geodesic} are called \textit{displacement geodesics}. They can be seen as the shortest paths from $\nu$ to $\mu$ in terms of mass transport (\cite{Santambrogio:2015} Theorem 5.27). It can be shown that there exists a vector field $(t,x)\mapsto v_t(x)$ with values in $\R^d$ such that $\rho$ satisfies the continuity equation \manote{reference} :
\begin{equation}\label{eq:continuity_equation}
\partial_t \rho_t + div(\rho_t v_t ) = 0 \qquad \forall t\in[0,1].
\end{equation}
Equation \cref{eq:continuity_equation} is well defined in distribution sense even when $\rho_t$ doesn't have a density. $v_t$ can be interpreted as a tangent vector to the curve $(\rho_t)_{t\in[0,1]}$ at time $t$ so that the length $l(\rho)$ of the curve $\rho$ would be given by:
\begin{equation}
l(\rho)^2 = \int_0^1 \Vert v_t \Vert^2_{L_2(\rho_t)} \diff t \quad \text{ where } \quad 
\Vert v_t \Vert^2_{L_2(\rho_t)} =  \int \Vert v_t(x) \Vert^2 \diff \rho_t(x)
\end{equation}
\aknote{add constant speed geodesics}
This perspective allows to provide a dynamical interpretation of the $W_2$ as the length  of the shortest path from $\nu$ to $\mu$ and is summarized by the celebrated Benamou-Brenier formula (\cite{Santambrogio:2015} 5.28 ):
\begin{align}\label{eq:benamou-brenier-formula}
W_2(\nu,\mu) = \inf_{(\rho,v)} l(\rho)
\end{align}
where the infimum is taken  over all couples  $\rho$ and $v$ satisfying  \cref{eq:continuity_equation}  with boundary conditions given by \cref{eq:boundary_conditions}.

\begin{remark}
	Such paths should not be confused with another kind of paths called \textit{mixture geodesics}. The mixture geodesic $(m_t)_{t\in[0,1]}$ from $\nu$ to $\mu$ is obtained by first choosing either $\nu$ or $\mu$ according to a Bernoulli distribution of parameter $t$ and then sampling from the chosen distribution:
	\begin{align}\label{eq:mixture_geodesic}
	m_t = (1-t)\nu + t\mu \qquad \forall t \in [0,1].
	\end{align}
	Paths of the form \cref{eq:mixture_geodesic} can be thought as the shortest paths between two distributions when distances on $\mathcal{P}_2(\X)$ are measured using the $MMD$ (\cite{Bottou:2017} Theorem 5.3). We refer to \cite{Bottou:2017} for an overview of the notion of shortest paths in probability spaces and for the differences between mixture geodesics and displacement geodesics.
	Although, we will be interested in the $MMD$ as a loss function, we will not consider the geodesics that are naturally associated to it and we will rather consider the displacement geodesics defined in \cref{eq:displacement_geodesic} for reason that will become clear in \cref{subsec:wasserstein_flow}.
\end{remark}


\subsubsection{Gradient flows on the space of probability measures}\label{sec:gradient_flows_functionals}


Let $\F : \mathcal{P}(\X) \rightarrow \R \cup \infty$, $\rho \mapsto \F(\rho)$ a functional. We call $\frac{\partial{\F}}{\partial{\rho}}$ if it exists, the unique (up to additive constants) function such that $\frac{d}{d\epsilon}\F(\rho+\epsilon  f)_{\epsilon=0}=\int\frac{\partial{\F}}{\partial{\rho}}(\rho) df$ for every perturbation $f$ such that, at least for $\epsilon \in [0, \epsilon_0]$, the measure $\rho +\epsilon f$ belongs to $\mathcal{P}(\X)$. The function $\frac{\partial{\F}}{\partial{\rho}}$ is called first variation of the functional $\F$ at $\rho$. Consider a \textit{Lyapunov functional} (or "free energy" or "entropy") $\F$ (see \citep{Villani:2004}), i.e. a functional of the form:
\begin{equation}\label{eq:lyapunov}
\F(\rho)=\int U(\rho(x)) \rho(x)dx + \int V(x)\rho(x)dx + \int W(x,y)\rho(x)\rho(y)dxdy
\end{equation}
where  $U$ is the internal energy, $V$ the potential energy and $W$ the
interaction energy. The formal gradient flow equation associated to this functional can be written:
\begin{equation}\label{eq:continuity_equation1}
\frac{\partial \rho}{\partial t}= div( \rho \nabla \frac{\partial \F}{\partial \rho})=div(\rho\nabla (U'(\rho)+V+W*\rho))
\end{equation}
where $\nabla \frac{\partial \F}{\partial \rho}$ is the strong subdifferential of $\F(\rho_t)$ associated with the 2-Wasserstein
metric (see \cite{ambrosio2008gradient}, Lemma 10.4.1). Indeed, for some generalized notion of gradient $\nabla_{W_2}$, and for sufficiently regular $\rho$ and $\F$, the r.h.s. of \eqref{eq:continuity_equation1} corresponds to $-\nabla_{W_2}\F(\rho)$.
And the dissipation of entropy is defined as: %see  http://wwwf.imperial.ac.uk/~jcarrill/RICAM/CharlaRICAM2014-1.pdf
\begin{align}
&        \frac{d \F(\rho)}{dt} =-D(\rho) \quad \text{ with } D(\rho)= \int |\xi|^2 \rho(x)dx\\
&\text{ and } \xi= \nabla (U'(\rho) + V + W * \rho)= \nabla \frac{\partial \F}{\partial \rho}
\end{align}
Standard considerations from fluid mechanics tell us that the continuity equation \eqref{eq:continuity_equation1} may be interpreted as the equation ruling the evolution of the density $\rho_t$ of a family of particles initially distributed according to some $\rho_0$ and each of which follows the velocity/vector field $v_t=\nabla \frac{\partial{\F}}{\partial{\rho_t}}$.

\begin{remark} \label{rem:KL_Lyapunov}
	A famous example of a functional \eqref{eq:lyapunov} is the Kullback-Leibler divergence, defined for $\rho, \mu \in \mathcal{M}^+$ by
	$KL(\rho,\mu)=\int log(\frac{\rho(x)}{\mu(x)})\rho(x)dx$. Indeed, $KL(\rho, \mu)=\int U(\rho(x))\rho(x)dx + \int V(x) \rho(x)dx$ with $U(\rho(x))=\rho(x)log(\rho(x))$ the entropy function and $V(x)=-log(\mu(x))$. In this case, $\nabla \frac{\partial \F}{\partial \rho}= \nabla \log(\rho) + \nabla V=  \nabla \log(\frac{\rho}{\mu})$ and equation \eqref{eq:continuity_equation} becomes the classical Fokker-Planck equation:
	\begin{equation}\label{eq:Fokker-Planck}
	\frac{\partial{\rho}}{\partial t}= div(\rho \nabla V )+ \Delta \rho
	\end{equation}
	It can be shown \aknote{ref? I think it's Jordan} that the distribution of the Langevin diffusion:
	\begin{equation}\label{eq:langevin_diffusion}
	dX_t= -\nabla \log \mu (X_t)dt+\sqrt{2}dB_t
	\end{equation}
	where $(B_t)_{t\ge0}$ is a $d$-dimensional Brownian motion, satisfies \eqref{eq:Fokker-Planck}.
\end{remark}



\subsection{Maximum Mean Discrepancy}\label{subsec:MMD}
For a given characteristic kernel $k$ defined on $\X$, we denote by $\kH$ its corresponding Reproducing Kernel Hilbert Space \manote{some reference here needed}. $\kH$ is a Hilbert space with inner product $\langle .,. \rangle_{\kH}$ and corresponding norm $\Vert . \Vert_{\kH}$. The unit ball in $\kH$ which will be denoted as $\mathcal{B}$ is simply the set of functions $f$ in $\kH$ such that $\Vert f\Vert_{\kH}\leq 1 $:
\begin{align}\label{eq:unit_ball_RKHS}
\mathcal{B} = \{ f\in \kH : \quad \Vert f\Vert_{\kH}\leq 1 \}
\end{align}
Under mild conditions \manote{write conditions} on the kernel $k$, it is possible to define a distance on $\mathcal{P}_2(\X)$ by finding a function $f$ in $\mathcal{B}$ that maximizes the mean difference between two given distributions $\mu$ and $\nu$. Such distance is called the Maximum Mean Discrepancy  (MMD) \cite{Gretton:2012}:
\begin{align}\label{eq:MMD}
MMD(\mu,\nu) = \sup_{g\in \mathcal{B}} \int g\diff\mu - \int g \diff\nu
\end{align}
The maximization problem in \cref{eq:MMD} is achieved for an optimal $g^*$ in $\mathcal{B}$ that is proportional to the  witness function between $\nu$ and $\mu$:
\begin{align}\label{eq:witness_function}
f_{\nu,\mu}(z) = \int k(.,z)\diff \mu - \int k(.,z)\diff \nu  \qquad z\in \X
\end{align}
This allows to express the $MMD$ as the norm of \cref{eq:witness_function} $f_{\nu,\mu}$:
\begin{align}\label{eq:mmd_norm_witness}
MMD(\mu,\nu) = \Vert f_{\nu,\mu} \Vert_{\mathcal{H}} 
\end{align}
Furthermore, a closed form expression in terms of expectations of the kernel under $\mu$ and $\nu$ can be obtained \cite{gretton2012kernel}:
\begin{align}\label{eq:closed_form_MMD}
MMD^2(\mu,\nu) = \int k\diff\mu \diff\mu + \int k\diff\nu \diff \nu - 2\int k\diff\mu \diff \nu
\end{align}
When samples from both $\mu$ and $\nu$ are available \cref{eq:closed_form_MMD} can be estimated using those samples. For a fixed target distributions $\mu$ we will consider the loss functional defined as:
\begin{align}\label{eq:loss_functional}
\F(\nu) = \frac{1}{2} MMD^2(\mu,\nu) \qquad \forall \nu \in \mathcal{P}_2(\X).
\end{align}
The next section describes the dynamics of the gradient flow of \cref{eq:loss_functional} under the $2$-Wasserstein metric as defined in \cref{subsec:wasserstein_flow}.
%The MMD was successfully used for training generative models (\cite{mmd-gan,Binkowski:2018,Arbel:2018}) where it is used in a loss functional to learn the parameters of the generator network. This motivate the  

\section{MMD Gradient flow - Algorithms, experiments}\label{sec:mmd_flow}

\subsection{MMD Descent - Theoretical algorithm}

We will consider a flow $(\rho_t)_{t>0}$ as described in \cref{sec:gradient_flows_functionals} and denote $f_t= \int k(.,z)\diff \mu - \int k(.,z)\diff \rho_t$. In this case:
\begin{equation}
\F(\rho_t)=\frac{1}{2}\|f_t\|^2_{\kH}
%&= \E_{\rho_t \otimes \rho_t}[k(X,X')]+\E_{\pi \otimes \pi}[k(Y,Y')] - 2\E_{\rho_t \otimes \pi}[k(X,Y)]
\end{equation} 

We define the potential energy (also called confinement energy) $V$ and interaction energy $W$ as follows:
\begin{equation}
V(X)=-\int 2 k(X,x')\mu(x')\text{,} \quad
W(X,Y)=k(X,Y)
\end{equation}
We have $MMD^2(\rho,\mu)=C+ \int V(x) \rho(x)dx + \int W(x,x')\rho(x)\rho(x')$, where $C=\E_{\mu\otimes \mu}[k(Y,Y')]$. $MMD^2$ can thus be written as a \textit{Lyapunov functional} (or "free energy" or "entropy") $\F$. \aknote{add that interestingly, both KL and MMD have the V potential term, but the diffusion of the particle derive from U for KL and from W for MMD?}


\begin{proposition}\label{prop:mmd_flow}
 The velocity in \eqref{eq:continuity_equation1} is given by $\nabla \frac{\partial{\F}}{\partial{\rho_t}}=2 \nabla f_t$ and the dissipation of MMD can be written:  
	\begin{equation}
	\frac{d MMD^2(\rho_t, \mu)}{dt}=-\E_{X \sim \rho_t}[\|\nabla f_t(X)\|^2]
	\end{equation}
	where $\nabla f_t(Y)= \int \nabla_{Y}k(.,Y) d\mu -  \int \nabla_{Y}k(.,Y) d\rho_t$.
\end{proposition}

\begin{remark}
	If the functional $\F$ was the KL divergence and $\rho_t$ a weak solution of the Fokker-Planck equation \eqref{eq:Fokker-Planck}, we would obtain the following dissipation (see \cite{wibisono2018sampling}):
	\begin{equation}
	\frac{d KL(\rho_t, \mu)}{dt}=-\E_{X \sim \rho_t}[\|\nabla log(\frac{\rho_t}{\mu}(X))\|^2]
	\end{equation}
\end{remark}


As explained in \cref{sec:gradient_flows_functionals} and according to \cref{prop:mmd_flow}, the gradient flow of the MMD can be written:
\begin{equation*}
\frac{\partial \rho_t}{\partial t}= div(\rho_t  \nabla f_t)
\end{equation*}
which is the density of the stochastic process (see \cref{sec:ito_stochastic}):
\begin{equation}\label{eq:stochastic_process}
dX_t=-\nabla f_t(X_t) 
\end{equation}
\eqref{eq:stochastic_process} represents the position $X_t$ of a particle at time $t > 0$.
We naturally consider the Euler discretization of \eqref{eq:stochastic_process}, which gives:
\begin{equation}\label{eq:discretized_process}
X_{k+1}=X_k - \gamma_{k+1} \nabla f_k(X_k)
\end{equation}
where $\nabla f_k(X_k)= \int \nabla_{X_k}k(X,X_k)\diff \mu - \int \nabla_{X_k}k(X,X_k) \diff \rho_k$ with $\rho_k$ the distribution of the process \eqref{eq:discretized_process} and $(\gamma_k)_{k\ge1}$ is a sequence of step sizes.




%In this subsection we assume that $MMD^2$ is $\lambda$-geodesically-convex. Conditions under which this holds will be provided in the next section.

%\subsection{Analysis of the theoretical algorithm}

%Equation~\eqref{eq:discretized_process} provides a theoretical algorithm to minimize $MMD^2(\cdot,\pi)$. The algorithm is only theoretical because it requires to compute $\nabla f_k(X_k)$.

%This algorithm is the discretization of the Gradient flow associated to $MMD^2$. Since $MMD^2$ is $\lambda$-convex, using Theorem 11.1.4 of~\cite{ambrosio2008gradient}, the gradient flow $(\rho_t)$ satisfies
%\begin{equation}
%    \label{eq:evi}
%    \frac12 \frac{d}{dt} W^2(\rho_t,\nu) + \frac{\lambda}{2}W^2(\rho_t,\nu) \leq MMD^2(\nu,\mu) - MMD^2(\rho_t,\pi)
%\end{equation}
%Unfortunately, $\lambda \leq 0$ (otherwise it would mean that $MMD^2$ is strongly-geodesically-convex and hence geodesically convex).

%For the theoretical algorithm~\eqref{eq:discretized_process} we can expect a discretized version of~\eqref{eq:evi} to hold : \asnote{This should hold, I haven't proved it yet but I will do it later}
%\begin{equation}
%    \label{eq:evi-discrete}
%    W^2(\rho_{k+1},\pi) \leq  W^2(\rho_{k},\pi) -2\gamma_{k+1}\left( \frac{\lambda}{2}W^2(\rho_{k},\pi) + MMD^2(\rho_{k+1},\pi) - MMD^2(\pi,\pi)\right)
%\end{equation}
%Since $\lambda \leq 0$ we cannot have a rate from this inequality (I think).
%However, if $\sum \gamma_k < \infty$, using Robbins Siegmund lemma, we know that $W^2(\rho_{k},\pi)$ converges to some $\ell \geq 0$. \asnote{From this it might be possible to prove that $W^2(\overline{\rho_{k}},\pi)$ converges to zero, but it would be a lot of work. It looks like Pakes Hasminskii criterion}

%\subsection{Another Lyapunov function}

%In this section we try to use $MMD^2(\cdot,\pi)$ as a Lyapunov function (instead of $W^2(\cdot,\pi)$), like in Theorem 3.3 of Liu 2017. Once this is done, we can use the Gradient Lojasiewicz inequality to get a rate (see Bolte, it's like log sobolev inequality)

%Taylor : 


\subsection{Space discretization - Sample-based setting}

Two settings are usually encountered in the sampling literature: one is called \textit{density-based}, i.e. $\mu$ is known up to a constant, and the other is called \textit{sample-based}, i.e. we only have access to a set of samples $X \sim \mu$.
The Unadjusted Langevin Algorithm (ULA) seems much more adapted to the first setting, since it only requires the knowledge of $\nabla \log \mu$, whereas our algorithm requires the knowledge of $\mu$ (since $\nabla f_t$ involves an integration over $\mu$). However, in the sample-based setting, it may be difficult to adapt the ULA algorithm, since it would require firstly to estimate $\nabla \log(\mu)$ based on a set of samples of $\mu$, before plugging this estimate in $\eqref{eq:langevin_algorithm}$. This problem, sometimes referred to as \textit{score estimation} in the literature, has been the subject of a lot of work but remains hard especially in high dimensions\aknote{be more precise}. In contrast, the gradient of $f_t$ can be 'easily'\aknote{complete by a rate on MMD} estimated by:
\begin{equation}
\widehat{\nabla f_t}(z)= \frac{1}{n}\sum_{i=1}^{n}\nabla_{z}k(u_i,z) -\frac{1}{n}\sum_{i=1}^{n}\nabla_{z}k(v_i,z) 
\end{equation}
where $(u_1, \dots, u_n)\sim \mu$ and $(v_1, \dots, v_n)\sim \rho_t$. We denote by $\widehat{ \mu}=\sum_{j=1}^{n}\delta_{u_i}$. Since we do not have access to $(v_1, \dots, v_n)\sim \rho_t$, we will propose a particle system.


\vspace{0.5cm}
The following is based on the formalism and some results of \cite{jourdain2007nonlinear}. Firstly, we recall that \eqref{eq:stochastic_process} can be written as a Mac-Kean Vlasov model, a particular kind of SDE driven by a Levy process:
\begin{align}\label{eq:theoretical_process}
&X_t=X_{0}+\int_{0}^t \sigma(X_s, \rho_s, \mu)ds \quad \text{for t in [0,T]}\\
&\forall s \in [0,T]\;,\quad \rho_s \text{ denotes the probability distribution of } X_s
\end{align}
with $\sigma(X_s, \rho_s, \mu)=-\nabla f_t(X_s)=\int \nabla_{X_s}k(.,X_s) d\rho_t -  \int \nabla_{X_s}k(.,X_s) d\mu$. Notice that $\sigma$ is Lipschitz continuous in its second and third variable.

In the sample-based setting, i.e. given $\widehat{\mu}$, it is thus natural to consider the following system of $n$ interacting particles:
\begin{align}\label{eq:sample_based_process}
&\widehat{X}_t^{j,n}=X_{0}+\int_{0}^t \sigma(\widehat{X}_s^{j,n}, \widehat{\rho}_s^n, \widehat{\mu})ds \quad \text{for t in [0,T]}\\
&\forall s \in [0,T]\;,\quad \widehat{\rho}_s^n=\sum_{j=1}^{n} \delta_{\widehat{X}_s^{j,n}} \text{ denotes the empirical measure } 
\end{align}



%It is thus natural to consider the stochastic process:
%\begin{equation}\label{eq:sample_based_process}
%dY_{t}=-\widehat{\nabla f_t}(Y_t) 
%\end{equation}


\begin{proposition}\aknote{proof not complete} Let $\rho_t$, $\widehat{\rho}_t$, the distributions of the processes \eqref{eq:theoretical_process} and \eqref{eq:sample_based_process} respectively. Suppose that $k$ is bounded and measurable on $\X$, and that there exists $L_k$ such that $\forall x,y \in \X$, $\| k(x,.)-k(y,.) \|_{\kH}\le L_k \|x-y\|$. Then:
	\begin{equation}
	MMD^2(\rho_t,\widehat{\rho_t})\le ?%L_k^2( \frac{C_1}{n}+ \frac{C_2}{n^{\frac{1}{d}}})
	\end{equation}
\end{proposition}
\begin{proof}  \aknote{true?}
	%We now introduce the process:
	%\begin{equation}\label{eq:intermediary_process}
	%dZ_{t}=\widetilde{\nabla f_t}(Z_t) 
	%\end{equation}
	%where $\widetilde{\nabla f_t}(z)=\int \nabla_{z}k(X,z)\diff \mu- \frac{1}{n}\sum_{i=1}^{n}\nabla_{z}k(v_i,z)$ and $(v_1, \dots, v_n)\sim \rho_t$.  Let $\widetilde{\rho}_t$ be the distribution of \eqref{eq:intermediary_process}. 
	Introduce the system of $n$ interacting particles:
	\begin{align}\label{eq:intermediary_process}
	&X_t^{j,n}=X_{0}+\int_{0}^t \sigma(X_s^{j,n}, \rho_s^n, \mu)ds \quad \text{for t in [0,T]}\\
	&\forall s \in [0,T]\;,\quad \rho_s^n=\sum_{j=1}^{n} \delta_{X_s^{j,n}} \text{ denotes the empirical measure } 
	\end{align}
	By Theorem 3 in \cite{jourdain2007nonlinear}, since $\sigma$ is Lipschitz continuous in its second variable, we have:	
	\begin{equation}\label{eq:upp_bound1}
	\sup_{j \le n}\E[\sup_{t\le T}\| X_t - X_t^{j,n}\|^2]\le \frac{C_1}{n}
	\end{equation}
	
		
	\vspace{2cm}\aknote{the following is a wrong proof. but some elements can be useful later}
	Let $\widetilde{\rho_t}$ be the distribution of \eqref{eq:intermediary_process}. We firstly have the following decomposition:
	\begin{align}\label{eq:decompose_process}
	MMD^2(\widehat{\rho_t}, \rho_t) \le MMD^2(\widehat{\rho_t}, \widetilde{\rho_t})+ MMD^2(\widetilde{\rho_t}, \rho_t)
	\end{align}
	 We will firstly bound the first term on the r.h.s. of \eqref{eq:decompose_process}. By \cref{lem:mmd_w2}, we have that:
	 \begin{equation}
	 MMD^2(\widehat{\rho_t}, \widetilde{\rho_t})\le L_k^2 W_2^2(\widehat{\rho_t}, \widetilde{\rho_t})
	 \end{equation}
	 Then:
	 \begin{align}
	 W_2^2(\rho_t, \widetilde{\rho_t}) = \inf_{\pi \in \Pi(\rho_t, \widetilde{\rho_t})} \int \| x-z \|^2 d\pi(\rho_t, \widetilde{\rho_t}) \le \int \|x-z\|^2 d\rho_t(x)d\widetilde{\rho_t}(z)
	 \end{align}
	 Hence by \eqref{eq:upp_bound1}:
	 \begin{equation}
	 MMD^2(\rho_t, \widetilde{\rho_t})\le L_k^2 \frac{C_1}{n}
	 \end{equation}
	We now turn to the second term at the r.h.s. of \eqref{eq:decompose_process}. We can derive a similar proof than Theorem 3 in \cite{jourdain2007nonlinear}:\aknote{same, we need Lipschitz continuity of the process in terms of the third variable}
	 \begin{align}
	 \E[\sup_{t\le T}\|Z_t-Y_t\|] \le   C \int_{O}^T |\sigma() -\sigma()| ds \le \E[W_1^2(\widehat{\mu},\mu)]
	 \end{align}
	where $\widehat{\mu}=\frac{1}{n}\sum_{i=1, \dots,n}\delta_{u_i}$ and $u_i \sim \mu$. It was shown in \cite{dudley1969speed} that when $d > 2$, if $\mu$ has a compact support in $\R^d$ then:
	\begin{equation}
		\E[W_1^2(\widehat{\mu},\mu)]\le \frac{C_2}{n^{\frac{1}{d}}}
	\end{equation}
	\begin{remark}
	Note that more recently, sharper rates of convergence  for $W_p(\widehat{ \mu}, \mu)$, for $p\ge 1$, have been computed in \cite{weed2017sharp} for a larger class of measures. These rates involve an intrinsic dimension of the measure $\mu$ (its Wassertein dimension). 
	\end{remark}
\end{proof}


\begin{remark}
	We point out here that algorithm~\eqref{eq:sample_based_process} is different from the descent proposed by \cite{mroueh2018regularized}. 
\end{remark}

\begin{remark}
	Birth-Death Dynamics to improve convergence (see \cite{rotskoff2019global}).
\end{remark}


\section{Discretizing the MMD gradient Flow}\label{sec:discretized_flow}



\subsection{Convergence of the time-discretized flow}


In all what follows we consider a fixed target distribution $\mu$ and define the following functional 
\begin{align}\label{eq:MMD_functional}
\nu \mapsto \F(\nu):=\frac{1}{2} MMD^2(\mu,\nu)\textbf{}
\end{align}
We investigate some theoretical properties of the MMD flow. In particular, we are interested in characterizing the convergence of the time discretized flow:
\begin{align}\label{eq:discretized_flow}
\nu_{n+1} = (I -\gamma \phi_n)_{\#}\nu_n
\end{align}
where $\gamma$ is some fixed step-size and $X \mapsto \phi_n(X):=\nabla f_{n}(X)$ is the gradient of the witness function between $\mu$ and $\nu_n$. It is easy to see that the particle version of \cref{eq:discretized_flow} is given by:
\begin{align}
X_{n+1} = X_n - \gamma \phi_n(X_n) \quad\forall n\in \mathbb{N}.
\end{align} 

\cref{prop:almost_convex_optimization} guarantees the existence of a direction of descent that minimizes the functional $\F$ provided that the starting point $\rho_1$ has a potential greater than the barrier $K$, i.e:
\begin{align}\label{eq:barrier_condition}
	\F(\rho_1)> \inf_{\rho\in \mathcal{P}} \F(\rho) + K
\end{align}
One natural question to ask is whether the  discretized gradient flow algorithm provides such way to reach the barrier $K$ and at what speed this happens. This subsection will answer that question. Firstly, we state few propositions that will lead us to the final result.


\begin{proposition}\label{prop:decreasing_functional}
	Under \cref{assump:bounded_trace,assump:bounded_hessian}, the following inequality holds:
	\begin{align*}
	\F(\nu_{n+1})-\F(\nu_n)\leq -\gamma (1-\frac{\gamma}{2}L )\int \Vert \phi_n(X)\Vert^2 d\nu_n
	\end{align*}
\end{proposition}

\begin{proposition}\label{prop:evi}
	Consider the sequence of distributions $\nu_n$ obtained from \cref{eq:discretized_flow}. If $\gamma \leq 1/L$, then
	\begin{align}
2\gamma(\F(\nu_{n+1})-\F(\mu))
\leq 
W_2^2(\nu_n,\mu)-W_2^2(\nu_{n+1},\mu)-2\gamma K(\rho^n).
\label{eq:evi}
\end{align}
where $(\rho^n_t)_{0\leq t \leq 1}$ is a constant-speed geodesic from $\nu_n$ to $\mu$ and $K(\rho^n):=\int_0^1 \Lambda(\rho^n_s,\dot{\rho^n}_s)(1-s)ds$.
\end{proposition}

\begin{theorem}\label{th:rates_mmd}
	Consider the sequence of distributions $\nu_n$ obtained from \cref{eq:discretized_flow}. If $\gamma \leq 1/L$, then
	\begin{align}
\F(\bar{\nu}_{n})-\F(\mu)\leq  \frac{W_2^2(\nu_0,\mu)}{2 \gamma n} -\bar{K}
\end{align}
where $\bar{\nu}=\frac{1}{N}\sum_{n=1}^N \nu_n$. Moreover, 
\begin{align}
\F(\nu_n)-\F(\mu)\leq  \frac{W_2^2(\nu_0,\mu)}{2 \gamma n} -\bar{K}.
\end{align}
\end{theorem}
\begin{proof}
Iterating in \cref{eq:evi} we get:
\begin{align}
	2\gamma \sum_{j=1}^{n+1} (\F(\nu_{j}) - \F(\mu)) \leq W_2^2(\nu_0,\mu) - 2\gamma \sum_{j=0}^n K(\rho^j)
\end{align}
Let us denote $\bar{K}$ the average value\asnote{$K(\rho^j)$ is bounded in $j$?} of $(K(\rho^j))_{0\leq j \leq n}$ over iterations from $0$ to $n$. Using \cref{lem:mixture_convexity} we have:
\begin{align}
\F(\bar{\nu}_{n+1})-\F(\mu)\leq  \frac{W_2^2(\nu_0,\mu)}{2 \gamma (n+1)} -\bar{K}
\end{align}
Now, consider the Lyapunov function $L_n = n \gamma (\F(\nu_n) - \F(\mu)) + \frac12 W_2^2(\nu_n,\mu)$. Then,
\begin{align*}
    L_{n+1} &= n\gamma(\F(\nu_{n+1}) - \F(\mu)) + \gamma(\F(\nu_{n+1}) - \F(\mu)) + \frac12 W_2^2(\nu_{n+1},\mu)\\
    &\leq n\gamma(\F(\nu_{n+1}) - \F(\mu)) + \frac12 W_2^2(\nu_n,\mu)-\gamma K(\rho^n)\\
    &\leq n\gamma(\F(\nu_{n}) - \F(\mu)) + \frac12 W_2^2(\nu_n,\mu)-\gamma K(\rho^n) -n\gamma^2 (1-\frac{\gamma}{2}L )\int \Vert \phi_n(X)\Vert^2 d\nu_n \\
    &\leq  L_n - \gamma K(\rho^n).
\end{align*}
where we used Proposition~\cref{prop:decreasing_functional} in the penultimate inequality\asnote{Les deux derniers termes pourraient-ils se manger par miracle?}.
Finally, 
\begin{equation}
    n\gamma (\F(\nu_{n}) - \F(\mu)) \leq L_n \leq L_0 -\gamma \sum_{j = 0}^{n-1} K(\rho^j)
\end{equation}
\end{proof}



%In this subsection we assume that $MMD^2$ is $\lambda$-geodesically-convex. Conditions under which this holds will be provided in the next section.

%\subsection{Analysis of the theoretical algorithm}

%Equation~\eqref{eq:discretized_process} provides a theoretical algorithm to minimize $MMD^2(\cdot,\pi)$. The algorithm is only theoretical because it requires to compute $\nabla f_k(X_k)$.

%This algorithm is the discretization of the Gradient flow associated to $MMD^2$. Since $MMD^2$ is $\lambda$-convex, using Theorem 11.1.4 of~\cite{ambrosio2008gradient}, the gradient flow $(\rho_t)$ satisfies
%\begin{equation}
%    \label{eq:evi}
%    \frac12 \frac{d}{dt} W^2(\rho_t,\nu) + \frac{\lambda}{2}W^2(\rho_t,\nu) \leq MMD^2(\nu,\mu) - MMD^2(\rho_t,\pi)
%\end{equation}
%Unfortunately, $\lambda \leq 0$ (otherwise it would mean that $MMD^2$ is strongly-geodesically-convex and hence geodesically convex).

%For the theoretical algorithm~\eqref{eq:discretized_process} we can expect a discretized version of~\eqref{eq:evi} to hold : \asnote{This should hold, I haven't proved it yet but I will do it later}
%\begin{equation}
%    \label{eq:evi-discrete}
%    W^2(\rho_{k+1},\pi) \leq  W^2(\rho_{k},\pi) -2\gamma_{k+1}\left( \frac{\lambda}{2}W^2(\rho_{k},\pi) + MMD^2(\rho_{k+1},\pi) - MMD^2(\pi,\pi)\right)
%\end{equation}
%Since $\lambda \leq 0$ we cannot have a rate from this inequality (I think).
%However, if $\sum \gamma_k < \infty$, using Robbins Siegmund lemma, we know that $W^2(\rho_{k},\pi)$ converges to some $\ell \geq 0$. \asnote{From this it might be possible to prove that $W^2(\overline{\rho_{k}},\pi)$ converges to zero, but it would be a lot of work. It looks like Pakes Hasminskii criterion}

%\subsection{Another Lyapunov function}

%In this section we try to use $MMD^2(\cdot,\pi)$ as a Lyapunov function (instead of $W^2(\cdot,\pi)$), like in Theorem 3.3 of Liu 2017. Once this is done, we can use the Gradient Lojasiewicz inequality to get a rate (see Bolte, it's like log sobolev inequality)

%Taylor : 


\subsection{Convergence of the space discretized flow - Sample-based setting}

Two settings are usually encountered in the sampling literature: one is called \textit{density-based}, i.e. $\mu$ is known up to a constant, and the other is called \textit{sample-based}, i.e. we only have access to a set of samples $X \sim \mu$.
The Unadjusted Langevin Algorithm (ULA), which involves a time-discretized version of \eqref{eq:langevin_diffusion}, seems much more adapted to the first setting, since it only requires the knowledge of $\nabla \log \mu$, whereas our algorithm requires the knowledge of $\mu$ (since $\nabla f_t$ involves an integration over $\mu$). However, in the sample-based setting, it may be difficult to adapt the ULA algorithm, since it would require firstly to estimate $\nabla \log(\mu)$ based on a set of samples of $\mu$, before plugging this estimate in the update of the algorithm. This problem, sometimes referred to as \textit{score estimation} in the literature, has been the subject of a lot of work but remains hard especially in high dimensions\aknote{be more precise}. In contrast, the gradient of $f_t$ can be 'easily'\aknote{complete by a rate on MMD} estimated by:
\begin{equation}
\widehat{\nabla f_t}(z)= \frac{1}{n}\sum_{i=1}^{n}\nabla_{z}k(u_i,z) -\frac{1}{n}\sum_{i=1}^{n}\nabla_{z}k(v_i,z) 
\end{equation}
where $(u_1, \dots, u_n)\sim \mu$ and $(v_1, \dots, v_n)\sim \rho_t$. We denote by $\widehat{ \mu}=\sum_{j=1}^{n}\delta_{u_i}$. In the sample-based setting, i.e. given $\widehat{\mu}$, since we do not have access to $(v_1, \dots, v_n)\sim \rho_t$, it is natural to consider the following system of $n$ interacting particles (sometimes referred to as \textit{mean-field interaction} in mathematical physics and stochastic analysis):
\begin{align}\label{eq:sample_based_process}
&\widehat{X}_t^{j,n}=X_{0}+\int_{0}^t \sigma(\widehat{X}_s^{j,n}, \widehat{\rho}_s^n, \widehat{\mu})ds \quad \text{for t in [0,T]}\\
&\forall s \in [0,T]\;,\quad \widehat{\rho}_s^n=\sum_{j=1}^{n} \delta_{\widehat{X}_s^{j,n}} \text{ denotes the empirical measure } 
\end{align}
The convergence of the empirical measure of the particle system \eqref{eq:sample_based_process} to the solution of \eqref{eq:theoretical_process} has been stated under the name propagation of chaos (see \cite{kac1956foundations}, \cite{sznitman1991topics}).
More precisely, we need a uniform propagation of chaos (i.e bounded uniformly in time).


%It is thus natural to consider the stochastic process:
%\begin{equation}\label{eq:sample_based_process}
%dY_{t}=-\widehat{\nabla f_t}(Y_t) 
%\end{equation}


\begin{proposition}\aknote{proof not complete} Let $\widehat{\rho}_t^N$, be the distributions of the process \eqref{eq:sample_based_process}. %Suppose that $k$ is bounded and measurable on $\X$, and that there exists $L_k$ such that $\forall x,y \in \X$, $\| k(x,.)-k(y,.) \|_{\kH}\le L_k \|x-y\|$. Then:
	\begin{equation}
	MMD^2(\rho_t,\widehat{\rho}_t^N)\le ?%L_k^2( \frac{C_1}{n}+ \frac{C_2}{n^{\frac{1}{d}}})
	\end{equation}
\end{proposition}
\begin{proof} 
	Introduce the process:
	\begin{align}\label{eq:intermediary_process}
	&\widetilde{X}_t=X_{0}+\int_{0}^t \sigma(\widetilde{X}_s, \widetilde{\rho}_s, \widehat{\mu})ds \quad \text{for t in [0,T]}\\
	&\forall s \in [0,T]\;,\quad \widetilde{\rho}_s \text{ denotes the probability distribution of } X_s
	\end{align}
	\begin{lemma}
		By \cite{durmus2018elementary}, we have a uniform in time propagation of chaos:
		\begin{equation}
		W_1(\widetilde{\rho}_t,\widehat{\rho}_t^N)\le \frac{A}{N}
		\end{equation}
	\end{lemma}
	Indeed, we verify the three assumptions. 
	Now, it is a bit trickier to control $X_t-\widetilde{X_t}$. Maybe we can do:
	\begin{equation}\label{eq:decomposition}
	W_1(\mu, \widetilde{\rho}_t)\le W_1(\mu, \widehat{\mu})+W_1(\widehat{\mu}, \widetilde{\rho_t})
	\end{equation}
	Indeed, firstly we control the first r.h.s. term in \eqref{eq:decomposition} since it was shown in \cite{dudley1969speed} that when $d > 2$, if $\mu$ has a compact support in $\R^d$ then:
	\begin{equation}
	\E[W_1^2(\widehat{\mu},\mu)]\le \frac{C_2}{n^{\frac{1}{d}}}
	\end{equation}
	\begin{remark}
		Note that more recently, sharper rates of convergence  for $W_p(\widehat{ \mu}, \mu)$, for $p\ge 1$, have been computed in \cite{weed2017sharp} for a larger class of measures. These rates involve an intrinsic dimension of the measure $\mu$ (its Wassertein dimension). 
	\end{remark}
	Then, for the second term in \eqref{eq:decomposition}, we can apply the rates of convergence for the time continuous flow, applied to the process $\widetilde{X_t}$.\aknote{if $\widehat{ \mu}$ satisfy the assumptions we may need to apply these rates}
\end{proof}

\section{Conclusion}

\section*{TO DO}

~\cref{sec:mmd_flow}.
\begin{itemize}
	\item Investigate sample-based algorithm
	\item Birth-Death Dynamics
	\item Experiments!!
\end{itemize}

~\cref{sec:theory}
\begin{itemize}
	\item Get rates in $\nu_n$ (change Lyapunov function)
	\item Hard: Refine the bounds, quantify more precisely $K(\rho_n)$
	\item Hard: Polyak Lojasewicz (PL) for MMD
\end{itemize}
everywhere: treat comments if possible

%\subsubsection*{References}
%\renewcommand\refname{\vskip -1cm}
%\bibliographystyle{apalike}
%\bibliography{biblio}

\printbibliography

\clearpage
\appendix
%\newpage



\section{Appendix}\label{sec:appendix}

\subsection{Additional mathematical background}
%In all what follows, $\X$ is a convex subset of $\R^d$ and $\mathcal{P}_2(\X)$ denotes the set of all probability distributions supported on $\X$ with finite second moment.
%For a given distributions $\nu\in\mathcal{P}_2(\X)$ and an integrable function $f$ under $\nu$, the expectation of $f$ under $\nu$ will be written either as $\nu(f)$ or $\int f \diff\nu$ depending on the context. 



\subsubsection{Stochastic processes}\label{sec:ito_stochastic}

Consider the Itô process, i.e. the stochastic process:
\begin{equation}
dX_t=g(X_t)dt.
\end{equation}
Let $f$ be a twice-differentiable scalar function, Itô's formula (see \cite{ito1951stochastic}) can be written:
\begin{equation}
df(X_t)=\nabla f(X_t).g(X_t)dt
\end{equation}
Let $\rho_t$ be the distribution of the process $X_t$. We have:
\begin{align}
\E[\frac{df}{dt}(X_t)]&= \E[\nabla f(X_t).g(X_t)]\\
\Longleftrightarrow \int f(X) \frac{d \rho_t}{dt}(X)&=-\int f(X)div(g(X)\rho_t(X))
\end{align}
where the second line is obtained by integrating by parts on both sides of the equality. Finally, the distribution $\rho_t$ verifies the continuity equation: 
\begin{equation}
\frac{d\rho_t}{dt}=div(g\rho_t)
\end{equation}


\begin{lemma}\label{lem:mmd_w2}
	 Suppose that $k$ is bounded and measurable on $\X$, and that there exists $L_k$ such that $\forall x,y \in \X$, $\| k(x,.)-k(y,.) \|_{\kH}\le L_k \|x-y\|$. Then for all $\mu, \nu$ in $\mathcal{P}(\X)$:
	\begin{equation}
	MMD^2(\mu,\nu)\le  L_k W_1^2(\mu,\nu) \le L_k W_2^2(\mu,\nu)
	\end{equation}
\end{lemma}
\begin{proof}
Let $\mu, \nu$ in $\mathcal{P}(\X)$. By Proposition 20 in \cite{sriperumbudur2010hilbert} we have:
\begin{equation}
	MMD(\mu, \nu)	 \le \inf_{\pi \in \Pi(\mu, \nu)} \int \| k(x,.)-k(y,.) \|_{\kH} d\pi(\mu, \nu)
\end{equation}

Hence:
\begin{align}
	MMD^2(\mu, \nu)	
	 \le (\inf_{\pi \in \Pi(\mu, \nu)} \int L_k \| x-y \| d\pi(\mu, \nu))^2
 \le L_k^2 W_1^2(\mu, \nu) \le L_k^2 W_2^2(\mu,\nu)
\end{align}
\end{proof}

\subsection{Proofs}

\subsubsection{Proof of \cref{prop:mmd_flow}}

In the case where $\F= \frac{1}{2} \|f_t\|^2_{\kH}$, by simple derivations we obtain:
\begin{equation}
 \nabla \frac{\partial\frac{1}{2} \|f_t\|^2_{\kH}}{\partial \rho_t}= \nabla \langle \frac{\partial f_t}{\partial \rho_t}, f_t \rangle_{\kH}= \nabla \langle \frac{\partial \E_{\rho_t}[k(Y,.)]}{\partial \rho_t}, f_t \rangle_{\kH}= \nabla \langle k(Y,.), f_t \rangle_{\kH}
\end{equation}
Then, by applying the reproducing property we have that:
\begin{equation}
\nabla \langle k(Y,.), f_t \rangle_{\kH}
= \nabla f_t(Y)
\end{equation}
where $\nabla f_t(Y)= \E_{X \sim \rho_t}[\nabla_{Y}k(X,Y)] -  \E_{X \sim \pi}[\nabla_{Y}k(X,Y)]$.

\subsubsection{Proof of \cref{prop:lambda_convexity} (Displacement convexity)}

We will firstly need the following lemma.

\begin{lemma}\label{lem:derivatives_witness}
	Let  $\mu$, $\nu_0$ and $\nu_1$ be three distributions in $\mathcal{P}_2(\X)$ and consider a displacement geodesic $(\rho_t)_{t\in[0,1]}$ between $\nu_0$ and $\nu_1$  defined by \cref{eq:displacement_geodesic} 
	and its corresponding velocity vector $(v_t)_{t\in [0,1]}$ as defined in \cref{eq:continuity_equation}. The following statements hold:
	\begin{enumerate}
		\item The first and second time derivatives of the witness function $f_{\mu,\rho_t}$ between $\mu$ and $\rho_t$ are well defined elements in $ \kH$ and are given by:
		\begin{align}\label{eq:derivatives_witness}
		\dot{f}_{\mu,\rho_t} = \int \nabla_1 k(x,.).v_t(x) \diff \rho_t(x); \qquad
		\ddot{f}_{\mu,\rho_t} = \int v_t(x)^T\nabla_1^2 k(x,.).v_t(x) \diff \rho_t(x)
		\end{align}
		where $ x \mapsto \nabla_1 k(x,z)$ and $x\mapsto \nabla_1^2 k(x,z)$ respectively denote the gradient and hessian of $x\mapsto k(x,z)$ for a fixed $z$ in $\X$.
		\item For all $g\in \kH$:
		\begin{align}\label{eq:inner_prod_deriative_witness}
		\langle g,\dot{f}_{\mu,\rho_t}\rangle_{\kH} = \int \nabla_1 g.v_t \diff \rho_t; \qquad
		\langle g,  \ddot{f}_{\mu,\rho_t}\rangle_{\kH} = \int v_t^T\nabla_1^2 g.v_t \diff \rho_t
		\end{align}
		\item The RKHS norms of $\dot{f}_{\mu,\rho_t}$ and $\ddot{f}_{\mu,\rho_t}$ satisfy:
		\begin{align}\label{eq:norm_derivative_witness}
		\Vert \dot{f}_{\mu,\rho_t}\Vert_{\kH}^2 = \langle v_t,C_{\rho_t} v_t \rangle_{L_2(\rho_t)}; \qquad  \Vert \ddot{f}_{\mu,\rho_t} \Vert\leq \lambda \Vert v_t \Vert^2_{L_2(\rho_t)}  
		\end{align}
		with $\lambda$ given by \cref{assump:bounded_fourth_oder} and $C_{\nu}$ defined in \cref{prop:lambda_convexity}. 
	\end{enumerate} 
\end{lemma}
\begin{proof}
	By definition of $\rho_{t}$:
	\[
	f_t(z)= \int k(x,z)\diff \mu(x) - \int k(s_t(x,y),z)\diff \pi(x,y)
	\]
	\manote{proof}
\end{proof}


\begin{proof}
To prove that $\nu\mapsto \F(\nu)$ is $\Lambda$-convex
we need to compute the second derivative $\ddot{\F}(\rho_{t})$
where $\rho_{t}$ is a displacement geodesic between two probability
distributions $\nu_{0}$ and $\nu_{1}$ as defined in \cref{eq:displacement_geodesic}. Such a minimizing geodesic always exists and can be written as $\rho_t = (s_t)_{\#}\pi$ with $s_t$ defined in \cref{eq:convex_combination} and $\pi$ is an optimal coupling between $\nu_0$ and $\nu_1$ (\cite{Santambrogio:2015}, Theorem 5.27). Moreover, we denote by $v_t$ the corresponding velocity vector as defined in \cref{eq:continuity_equation}. Recall from \cref{eq:mmd_norm_witness} that $\F(\rho_t) = \frac{1}{2} \Vert f_{\mu,\rho_t}\Vert^2_{\mathcal{H}}$, with $f_{\mu,\rho_t}$ defined in \cref{eq:witness_function}. To simplify notations we will write $f_t:= f_{\mu,\rho_t}$. We start by computing the first derivative of $ t\mapsto \F(\rho_t) $. By  \cref{lem:derivatives_witness},\cref{eq:derivatives_witness}, we know that $\dot{f}_t$ and $\ddot{f}_t $ are well defined elements of $\kH$ for any given $t\in [0,1]$, hence 
\[
 \dot{\F}(\rho_t) = \langle f_t, \dot{f_t}\rangle_{\kH};\qquad \ddot{\F}(\rho_t) = \Vert \dot{f_t}\Vert^2_{\kH} + \langle f_t, \ddot{f_t}\rangle_{\kH}.
 \]
While $\Vert \dot{f_t}\Vert^2_{\kH}$ is non-negative, $\langle f_t, \ddot{f_t}\rangle_{\kH}$ can in general be negative. We are only interested in quantifying how negative it can get, for this purpose we use Cauchy-Schwartz inequality which directly gives:
\[
\ddot{\F}(\rho_t)\geq  \Vert \dot{f}_t \Vert^2_{\kH} - \Vert f_t \Vert_{\kH}\Vert \ddot{f}_t\Vert_{\kH} 
\]

Finally by \cref{lem:derivatives_witness}, \cref{eq:norm_derivative_witness}, we can conclude that:
\[
	\ddot{\F}(\rho_t)\geq  \langle v_t,(C_{\rho_t} - \lambda \F(\rho_t)^{\frac{1}{2}}) v_t \rangle_{L_2(\rho_t)} 
\]
with $C_{\rho_t}$ given by \cref{eq:positive_operator_C} and $I$ is the identity operator in $L_2(\rho_t)$. Now we can introduce the function:
\begin{align}
	\Lambda(\nu,v) = \langle v ,( C_{\nu} -\lambda \F(\nu)^{\frac{1}{2}} I) v \rangle_{L_2(\nu)} 
\end{align}
which is defined for any pair $(\nu,v)$ with  $\nu\in \mathcal{P}_2(\X)$ and $v$ a square integrable vector field in $L_2(\nu)$. It is clear that $\Lambda(\nu,.)$  is a quadratic form on $L_2(\nu)$. Therefore, from \cref{def:lambda-convexity} of $\Lambda$ convexity, we conclude that $\F$ is $\Lambda$-convex.
\end{proof}

%
%
%
%By \cref{lem:derivatives_witness}, we have that $\dot{f_t}\in \kH$ and 
%
% it follows from \manote{some assumption to exchange orders}
%\[
%\frac{df_{t}}{dt}=\int(\nabla\phi(x)-x).\nabla k(\pi_{t}(x),.)\nu_{0}(x)dx
%\]
%hence:
%\[
%\frac{dMMD^{2}(\mu,\rho_{t})}{dt}=2\int(\nabla\phi(x)-x).\nabla f_{t}(\pi_{t}(x))\nu_{0}(x)dx
%\]
%Now the second derivative is given by:
%\begin{align*}
%\frac{d^{2}MMD^{2}(\mu,\rho_{t})}{dt^{2}}= & \int(\nabla\phi(x)-x).Hf_{t}(\pi_{t}(x))(\nabla\phi(x)-x)\nu_{0}(x)dx\\
% & +\int(\nabla\phi(x)-x).\nabla_{1}\nabla_{2}k(\pi_{t}(x),\pi_{t}(x'))(\nabla\phi(x')-x')\nu_{0}(x)\nu_{0}(x')dxdx'
%\end{align*}
%Here $\nabla_{1}\nabla_{2}k(x,x')$ is the matrix whose components
%are given by $\langle\partial_{i}k(x,.),\partial_{j}k(x,.)\rangle$
%for $1\leq i,j\leq d$, and $Hf_{t}$ is the hesssian of $f_{t}$
%and its components are also given by:
%\[
%(Hf_{t}(x))_{i,j}=\langle f_{t},\partial_{i}\partial_{j}k(x,.)\rangle.
%\]
%Denoting by $h(x):=\nabla\phi(x)-x$ it follows that:
%\begin{align*}
%\frac{d^{2}MMD^{2}(\mu,\rho_{t})}{dt^{2}}= & \langle f_{t},\int\sum_{i,j}h_{i}(x)h_{j}(x)\partial_{i}\partial_{j}k(\pi_{t}(x),.)\nu_{0}(x)dx\rangle\\
% & +\Vert\int\sum_{i}h_{i}(x)\partial_{i}k(\pi_{t}(x),.)\nu_{0}(x)dx\Vert^{2}
%\end{align*}
%Now we use Cauchy-Schwartz inequality for the first term to get:
%\begin{align*}
%\frac{d^{2}MMD^{2}(\mu,\rho_{t})}{dt^{2}}\geq & -\Vert f_{t}\Vert_{\kH}\Vert\int\sum_{i,j}h_{i}(x)h_{j}(x)\partial_{i}\partial_{j}k(\pi_{t}(x),.)\nu_{0}(x)dx\Vert_{\kH}\\
% & +\Vert\int\sum_{i}h_{i}(x)\partial_{i}k(\pi_{t}(x),.)\nu_{0}(x)dx\Vert^{2}.
%\end{align*}
%After applying a change of variables $x=\pi_{t}(y)$ one recovers the
%velocity vector $v_{t}$ instead of $h$: 
%\begin{align*}
%\frac{d^{2}MMD^{2}(\mu,\rho_{t})}{dt^{2}}\geq & -\Vert f_{t}\Vert_{\kH}\Vert\int\sum_{i,j}v_{t}^{i}(x)v_{t}^{j}(x)\partial_{i}\partial_{j}k(x,.)\rho_{t}(x)dx\Vert_{\kH}\\
% & +\Vert\int\sum_{i}v_{t}^{i}(x)\partial_{i}k(x,.)\rho_{t}(x)dx\Vert^{2}.
%\end{align*}
%
%One can further note that:
%\[
%\Vert\int\sum_{i,j}v_{t}^{i}(x)v_{t}^{j}(x)\partial_{i}\partial_{j}k(x,.)\rho_{t}(x)dx\Vert_{\kH}\leq\lambda\Vert v_{t}\Vert_{L_{2}(\rho_{t})}^{2}
%\]
%
%and that 
%\begin{align*}
%\Vert\int\sum_{i}v_{t}^{i}(x)\partial_{i}k(x,.)\rho_{t}(x)dx\Vert^{2} & =\int v_{t}(x)^{T}\int\nabla_{1}\nabla_{2}k(x,x')v_{t}(x')\rho_{t}(x')dx'dx.\\
% & =\langle v_{t},C_{\rho_{t}}v_{t}\rangle_{L_{2}(\rho_{t})}
%\end{align*}
%
%Hence we have shown that 
%\[
%\frac{d^{2}MMD^{2}(\mu,\rho_{t})}{dt^{2}}\geq\langle v_{t},(C_{\rho_{t}}-\lambda MMD(\mu,\rho_{t})I)v_{t}\rangle_{L_{2}(\rho_{t})}=\Lambda(\rho_{t},v_{t})
%\]


\subsubsection{Proof of \cref{th:rates_mmd}}

\begin{lemma}	\label{lem:grad_flow_lambda_version}
Let $\nu$ be a distribution in $\mathcal{P}_2(\X)$ and $\mu$ the target distribution such that $\F(\mu)=0$.  Let $\pi$ be an optimal coupling between $\nu$ and $\mu$, and $\rho_t$ the displacement geodesic defined by \cref{eq:displacement_geodesic} with its corresponding velocity vector  $v_t$ as defined in \cref{eq:continuity_equation}. Finally let $\phi(X)=\nabla f_{\nu,\mu}(X)$ the gradient of the witness function between $\mu$ and $\nu$. The following inequality holds: \manote{This should be a standard result, just need to cite it}
\begin{align*}
	\int \phi(x).(y-x) d\pi(x,y)
	\leq
	\F(\mu)- \F(\nu) -\int_0^1 \Lambda(\rho_s,v_s)(1-s)ds
\end{align*}

\end{lemma}
\begin{proof}
Recall that $\rho_t$ is given by $\rho_t = (s_t)_{\#}\pi$. By $\Lambda$-convexity of $\mathcal{F}$ the following inequality holds:
	\begin{align*}
		\mathcal{F}(\rho_{t})\leq (1-t)\mathcal{F}(\nu)+t \mathcal{F}(\mu) - \int_0^1 \Lambda(\rho_s,v_s)G(s,t)ds
	\end{align*}
	Hence by bringing $\mathcal{F}(\nu)$ to the l.h.s and dividing by $t$ and then taking its limit at $0$ it follows that:
	\begin{align*}
	\dot{\F}(\rho_t)\vert_{t=0}\leq \mathcal	{F}(\mu)-\mathcal{F}(\nu)-\int_0^1 \Lambda(\rho_s,v_s)(1-s)ds.	
	\end{align*}
	Moreover, by \cref{lem:derivatives_witness}, the time derivative of the witness function between $\nu$ and $\mu$ is well defined, so that $\dot{\F}(\rho_t)$ can be written as:
	\[
	\dot{\F}(\rho_t) = \langle f_{\mu,\rho_t},\dot{f}_{\mu,\rho_t} \rangle_{\kH}
	\]
	Now by \cref{lem:derivatives_witness},\cref{eq:inner_prod_deriative_witness} it follows that:
\[
\dot{\F}(\rho_t) = \int \nabla f_{\mu,\rho_t}(x).v_t(x)\diff \rho_t(x)
\]
By definition of $\rho_t$,  one can further write:
\[
\dot{\F}(\rho_t) = \int \nabla f_{\mu,\rho_t}(s_t(x,y)).(y-x)\diff \pi(x,y)
\]
where we used the fact that $v_t(s_t(x,y))=(y-x)$\manote{cite something}. Hence at $t=0$ we get:
\[
\dot{\F}(\rho_t)\vert_{t=0} = \int \nabla f_{\mu,\nu}(x).(y-x)\diff \pi(x,y)
\]
which shows the desired result.
\end{proof}





\begin{lemma}\label{lem:derivative_mmd}\manote{Notations still needs to be adjusted in this lemma}
	Let $\phi$ be a vector field on $\X$ and $\nu$ in $\mathcal{P}_2(\X)$. Consider the path $\delta_t$ between $\nu$ and $(I+\phi)_{\#}\nu$ given by:
	\begin{align*}
		\delta_t=  (I+t\phi)_{\#}\nu \qquad \forall t\in [0,1]
	\end{align*}
The time derivative of $\mathcal{F}(\delta_t)$ is given by:
	\begin{align*}
		\dot{\F}(\delta_t)&=\int \nabla f_{\mu,\delta_t}(x+t\phi(x)) \phi(x)d\nu(x)\\
	\end{align*}
where $f_{\mu,\delta_t}$ is the witness function between $\mu$ and $\delta_t$ as defined in \cref{eq:witness_function}.	
	Moreover, under \cref{assump:bounded_trace,assump:bounded_hessian}, the second time derivative satisfies:
	\begin{align*}
		\ddot{\F}(\delta_t) \vert \leq 3L \int \Vert \phi(x) \Vert^2 d\nu(x)
	\end{align*}
	where $L$ is a positive constant defined in \cref{assump:bounded_trace,assump:bounded_hessian}.
	
\end{lemma}
\begin{proof}
For simplicity, we write $f_t$ instead of $f_{\mu,\delta_t}$.
We start by computing the first derivative. Recalling that $\mathcal{F}(\delta_t)$ is given by $\frac{1}{2}\Vert f_t\Vert^2_{\kH} $, it follows that:
\[
\dot{\F}(\delta_t)=\langle f_{t},\frac{df_{t}}{dt}\rangle_{\kH}.
\]
Using the definition
of $\delta_{t}=(I+t\phi)_{\#}\nu$ we can write:\aknote{$\pi_t$? guess this corresponds to the paragraph below}
\[
\frac{df_{t}}{dt}=\int \phi(X).\nabla k(\pi_{t}(X),.)d\nu(X),
\]
hence:
\[
\frac{d\mathcal{F}(\delta_{t})}{dt}=2\int\phi(X).\nabla f_{t}(\pi_{t}(X))d\nu(X)
\]
Now the second derivative is obtained by direct derivation of the above expression:
	\begin{align*}
		\frac{d^2 \mathcal{F}(\delta_t)}{dt^2} =& \int \phi(X)^THf_t(\pi_t(X))\phi(X)d\nu(X)\\ 
		&+\int \phi(X)^T\nabla_x\nabla_y k(\pi_t(X),\pi_t(X')) ) \phi(X')d\nu(X)d\nu(X') 
	\end{align*}
where $Hf_t$ is the hessian of $f_t$ in space and  $\nabla_x\nabla_y k(x,y)$ is the cross diagonal term of the hessian of $k$. By \ref{assump:bounded_hessian}, the first term in the above equation can be easily upper-bounded by:
\begin{align*}
	4L \int \Vert \phi(X)\Vert^2d\nu(X)  
\end{align*}
The last term can also be upper-bounded by $2L$ by \ref{assump:bounded_trace}.
\end{proof}

Let $  \nu$ and $\nu'$ be two distributions and $\Pi$ a coupling between $\nu$ and $\nu'$. We consider the path $\rho_t$ defined as $\rho_t=(\pi_t)_{\#}\Pi$ where $\pi_t(X,Y)=(1-t)X+tY$. It is possible to provide an expression for the time derivative of $\mathcal{F}{\rho_t}$. This is given by ?\\

%\begin{lemma}\label{lem:time_derivative}
%The time derivative of $\mathcal{F}(\rho_t)$ is given by:
%	\begin{align*}
%		\frac{d \mathcal{F}(\rho_t)}{dt}&=\int \nabla f_t(\pi_t(X)).(Y-X)d\Pi(X,Y)\\
%	\end{align*}
%	where $f_t$ is the witness function at time $t$ and is given by:
%	\begin{align}
%	f_t(x)=\rho_t(k(X,x))-\mu(k(X,x)) \qquad \forall t\in [0,1]
%	\end{align}	
%\end{lemma}
%\begin{proof}
%	The proof is very similar to the one in \cref{lem:derivative_mmd}. Indeed we still have
%	\begin{align*}
%		\frac{d \mathcal{F}(\rho_t)}{dt} = \langle f_t , \frac{df_t}{dt} \rangle
%	\end{align*}
%	And the time derivative of $f_t$ at each point $x\in\mathbb{R}^d$ is obtained by direct computation:
%	\begin{align*}
%		 \frac{df_t}{dt}= \int \nabla k(\pi_t(X,Y),.).(Y-X)d\Pi(X,Y)
%	\end{align*}
%	The result follows using the reproducing property in $\kH$.
% \end{proof}






\begin{proposition}\label{prop:decreasing_functional}
	Under \cref{assump:bounded_trace,assump:bounded_hessian}, the following inequality holds:
	\begin{align*}
	\F(\nu_{n+1})-\F(\nu_n)\leq -\gamma (1-\frac{\gamma}{2}L )\int \Vert \phi_n(X)\Vert^2 d\nu_n
	\end{align*}
\end{proposition}

\begin{proof}
	
	Here we consider a path between $\nu_n$ and $\nu_{n+1}$ of the form:
	\begin{align*}
	\rho_t	=(I-\gamma t\phi_n)_{\#}\nu_n
	\end{align*}
	The function $t\mapsto \mathcal{F}(\rho_t)$ is twice differentiable, hence one can use a Taylor expansion with integral remainder to get:
	\begin{align}\label{eq:taylor_expansion}
	\mathcal{F}(\nu_{n+1})-\mathcal{F}(\nu_{n})=\mathcal{F}(\rho_1)-\mathcal{F}(\rho_0) = \frac{d \mathcal{F}(\rho_t) }{dt}\vert_{t=0}+ \frac{1}{2} \int_0^1 \frac{d^2 \mathcal{F}(\rho_t)}{dt^2}(1-t)^2 dt 
	\end{align} 
	By taking $\phi=-\gamma \phi_n$ in \cref{lem:derivative_mmd} we have that:
	\begin{align*}
	\frac{d \mathcal{F}(\rho_t) }{dt} = -\gamma \int \nabla f_n(X).\phi_n(X)d\nu_n(X)=-\gamma \int \Vert \phi_n(X) \Vert^2 d\nu_n(X)
	\end{align*}
	since $\nabla f_n=\phi_n$.
	Moreover, by \cref{assump:bounded_trace,assump:bounded_hessian} it follows from \cref{lem:derivative_mmd} that:
	\begin{align}\label{eq:upper_bound_1}
	\vert \frac{d^2 \mathcal{F}(\rho_t) }{dt^2}   \vert\leq L\int \Vert \phi_n(X) \Vert^2 d\nu_n(X)
	\end{align}
	Using \cref{eq:taylor_expansion,eq:upper_bound_1} the result follows.
\end{proof}

\vspace*{1cm}

\begin{lemma}\label{lem:mixture_convexity}
	The functional $\F$ is mixture convex: for any probability distributions $\nu_1$ and $\nu_2$ and scalar $1\leq \lambda\leq 1$:
	\begin{align*}
	\F(\lambda \nu_1+(1-\lambda)\nu_2)\leq \lambda \F(\nu_1)+ (1-\lambda)\F(\nu_2)
	\end{align*}
\end{lemma}
\begin{proof}
	Let $\nu$ and $\nu'$ be two probability distributions and $0\leq \lambda\leq 1$.
	We need to show that \[\mathcal{F}(\lambda \nu + (1-\lambda)\nu') -\lambda \mathcal{F}(\nu) -(1-\lambda)\mathcal{F}(\nu')\leq 0\]
	This follows from a simple computation which shows that:\aknote{I'm not getting the same result}
	\begin{align*}
	\mathcal{F}(\lambda \nu + (1-\lambda)\nu') -\lambda \mathcal{F}(\nu) -(1-\lambda)\mathcal{F}(\nu') = -\frac{1}{2}\lambda(1-\lambda)MMD(\nu,\nu')^2 \leq 0.
	\end{align*}
\end{proof}


\subsection{Lojasiewicz type inequality}

Here we would like to derive an inequality between the time derivative of the Lyapounov functional $\mathcal{F}$ along its gradient flow $t\mapsto \nu_t$. For this purpose we first introduce the weighted negative Sobolev distance \manote{cite villani and peyre and Mroueh}:

\begin{align}\label{eq:neg_sobolev}
	\Vert \nu - \mu \Vert_{\dot{H}^{-1}(\nu)} = \sup_{\substack{ f\in W_0^{1,2}(\nu) \\ \nu(\Vert \nabla f \Vert^2) \leq 1 }} \vert \nu(f)-\mu(f)\vert 
\end{align}
Where $W_0^{1,2}(\nu)$ is the space $1$ order Sobolev functions with functions vanishing at the boundary of the domain.
The distance defined in \cref{eq:neg_sobolev} plays a fundamental role in dynamic optimal transport as it linearizes the $W_2$ distance when $\mu$ is arbitrarily close to $\nu$. It can also be seen as the minimum kinetic energy needed to advect the mass $\nu$ to $\mu$. However, this quantity might be infinite \manote{say exactly when it is finite} and one of the key problems would be to control its value during the evolution of the flow. More precisely we will rely on the following statement:
\begin{align}\label{eq:bounded_neg_sobolev}
	\Vert \nu_t  - \mu \Vert_{\dot{H}^{-1}(\nu_t)} \leq C \qquad \forall t\geq 0.
\end{align} 
where $\nu_t$ is defined by the gradient flow and $\mu$ is the target distribution. When \cref{eq:bounded_neg_sobolev}  holds, we have the following proposition:
\begin{proposition}\label{prop:PL_type_inequality}
	When \cref{eq:bounded_neg_sobolev} holds, the following inequality is then satisfied at all times:
	\begin{align}\label{eq:PL_type_inequality}
		\Vert \nabla f_t \Vert_{L_2(\nu_t)} \geq \frac{1}{C} \Vert f_t \Vert^2_{\mathcal{H}} \qquad \forall t\geq 0.
	\end{align}
\end{proposition}
\begin{proof}
	Indeed, this follows simply from the definition of the negative Sobolev distance: Consider $g = \Vert \nabla f_t\Vert^{-1}_{L_2(\nu_t)} f_t$, then $g\in W_0^{1,2}(\nu)$ \manote{this suggests an assumption on the kernel so that all those function satisfy a boundary condition} and $\Vert \nabla g \Vert_{L_2(\nu_t)}\leq 1$. Therefore, we directly have:
	\begin{align}
		\Vert \nu_t - \mu\Vert_{\dot{H}^{-1}(\nu_t)}\geq \vert \nu_t(g) - \mu(g)  \vert.
	\end{align}
Now, recall the definition of $g$, which implies that
\[
\vert \nu_t(g) - \mu(g)  \vert = \Vert \nabla f_t\Vert^{-1}_{L_2(\nu_t)} \vert \nu_t(f_t)-\mu(f_t)\vert.
\]
But since $f_t$  is exactly the witness functions between $\nu_t$ and $\mu$, it follows that $\nu_t(f_t)-\mu(f_t) = \Vert f_t\Vert^2_{\kH}$.
Using \cref{eq:bounded_neg_sobolev}, we get the desired inequality.
\end{proof}

Now we will use the inequality in \cref{prop:PL_type_inequality} to prove a convergence result towards the global optimum $\mu$. This is provided in \cref{prop:convergence}.
\begin{proposition}
$$
\cF(\nu_t) \leq \sqrt{\frac{C\cF(\nu_0)}{t}} 
$$
\end{proposition}
\begin{proof}
Consider the Lyapunov function $L(t) = t \cF(\nu_t)^2 + C \cF(\nu_t)$. Using Proposition~\ref{prop:mmd-flow} we know that $$C \frac{d}{dt} \cF(\nu_t) \leq - \cF(\nu_t)^2.$$
Hence, $t \mapsto \cF(\nu_t)$ is non increasing. Moreover it is non negative, hence $\frac{d}{dt}\cF(\nu_t)^2 \leq 0$. Then $$\frac{d}{dt}L(t) \leq \cF(\nu_t)^2 + C \frac{d}{dt}\cF(\nu_t) \leq 0.$$ 
Finally, $$t \cF(\nu_t)^2 \leq L(t) \leq L(0) = C \cF(\nu_0)$$
\end{proof}

\begin{proposition}\label{prop:convergence}
	If \cref{eq:bounded_neg_sobolev} is satisfied for all times then $t\mapsto \mathcal{F}(\nu_t)$ converges to $0$ with a rate of convergence given by:
	\begin{align}
		\mathcal{F}(\nu_t)\leq \frac{1}{\mathcal{F}(\nu_0)^{-1} + \frac{4t}{C}}
	\end{align}
\end{proposition}
\begin{proof}
	The proof is a simple consequence of \cref{prop:mmd_flow,eq:bounded_neg_sobolev}. Indeed, by \cref{prop:mmd_flow} we have that 
	\begin{align}
		\dot{\F}(\nu_t) = - \Vert \nabla f_t \Vert^2_{L_2(\nu_t)} 	
	\end{align}
	Using \cref{eq:PL_type_inequality}, we directly get that:
	\begin{align}
		\dot{\F}(\nu_t) \leq  -\frac{4}{C}\F(\nu_t)^2
	\end{align}
It is clear that if $\mathcal{F}(\nu_0)>0$ then $\F(\nu_t)>0$ at all times by uniqueness of the solution. Hence, one can divide by $\F(\nu_t)$ and integrate the inequality from $0$ to some time $t$. The desired inequality is obtained by simple calculations.
\end{proof}

All the difficulty is to see now when \cref{eq:bounded_neg_sobolev} holds. One possible strategy would be to start from initial $\nu_0$ such that $\Vert \nu_0  - \mu \Vert_{\dot{H}^{-1}(\nu_0)} \leq C $  for some finite positive value $C$ and then show that this property is preserved during the dynamics. It is also possible to have a time depended constant $C_t$ as long as its growth is such that:
\begin{align}
	\lim_{t\rightarrow +\infty} \int_0^t C_s^{-1}\diff s = +\infty
\end{align}
For instance $C_t$ could have up to a linear growth in time. In this case the decay of $\F(\nu_t)$ will no longer be in $\frac{1}{t}$ but only in $\frac{1}{\log(t)}$ \manote{This seems unlikely if we end up having convergence of $\nu_t$, but who nows.}.
One possible promising condition for \cref{eq:bounded_neg_sobolev} to hold would be if $\mu \ll \nu_0$ and if this property is preserved during the dynamics.


 










\end{document}
