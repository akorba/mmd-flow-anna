
\subsection{A rate of convergence for the continuous flow}\label{sec:Lojasiewicz_inequality}

Since we know by \cref{prop:decay_mmd} that Lyapunov functional $\mathcal{F}$ is decreasing along its gradient flow $t\mapsto \rho_t$, a natural question is to derive a rate of convergence.
%  Here we would like to derive an inequality between the time derivative of the Lyapunov functional $\mathcal{F}$ along its gradient flow $t\mapsto \rho_t$. 
For this purpose we first introduce the weighted negative Sobolev distance (see \cite{peyre2019computational}):% \manote{cite villani and peyre and Mroueh}:
\begin{align}\label{eq:neg_sobolev}
	\Vert \nu - \mu \Vert_{\dot{H}^{-1}(\nu)} = \sup_{\substack{ f\in W_0^{1,2}(\nu), \; \nu(\Vert \nabla f \Vert^2) \leq 1 }} \vert \nu(f)-\mu(f)\vert 
\end{align}
Where $W_0^{1,2}(\nu)$ is the space $1$ order Sobolev functions with functions vanishing at the boundary of the domain.
The distance defined in \cref{eq:neg_sobolev} plays a fundamental role in dynamic optimal transport as it linearizes the $W_2$ distance when $\mu$ is arbitrarily close to $\nu$. It can also be seen as the minimum kinetic energy needed to advect the mass $\nu$ to $\mu$ (see \cite{mroueh2018regularized}). More precisely we will rely on the following statement to derive a rate of convergence:
\begin{align}\label{eq:bounded_neg_sobolev}
	\Vert \nu_t  - \mu \Vert_{\dot{H}^{-1}(\nu_t)} \leq C \qquad \forall t\geq 0.
\end{align} 
where $\nu_t$ is defined by the gradient flow \eqref{eq:continuity_mmd} and $\mu$ is the target distribution. When \cref{eq:bounded_neg_sobolev}  holds, we have the following proposition:
\begin{proposition}\label{prop:lojasiewicz}
	When \cref{eq:bounded_neg_sobolev} holds, the following Lojasiewicz-type inequality is then satisfied:% at all times:
	\begin{align}\label{eq:PL_type_inequality}
		\Vert \nabla f_{\nu_t,\mu} \Vert_{L_2(\nu_t)} \geq \frac{1}{C} \Vert f_{\nu_t,\mu} \Vert^2_{\mathcal{H}} \qquad \forall t\geq 0.
	\end{align}
	Then $t\mapsto \mathcal{F}(\nu_t)$ converges to $0$ with a rate of convergence given by:
	\begin{align}
	\mathcal{F}(\nu_t)\leq \frac{1}{\mathcal{F}(\nu_0)^{-1} + \frac{4t}{C}}
	\end{align}
\end{proposition}

All the difficulty is to see now when \cref{eq:bounded_neg_sobolev} holds. However, this quantity might be infinite \manote{say exactly when it is finite} and a fundamental problem would be to control its value during the evolution of the flow. One possible strategy would be to start from initial $\nu_0$ such that $\Vert \nu_0  - \mu \Vert_{\dot{H}^{-1}(\nu_0)} \leq C $  for some finite positive value $C$ and then show that this property is preserved during the dynamics. For instance, it would be satisfied if $\mu \ll \nu_0$ and if this remains true along the flow. Another view on the problem would be to consider a time depended constant $C_t$ in \cref{eq:bounded_neg_sobolev}, such that $	\lim_{t\rightarrow +\infty} \int_0^t C_s^{-1}\diff s = +\infty$.
%\begin{align}
%	\lim_{t\rightarrow +\infty} \int_0^t C_s^{-1}\diff s = +\infty
%\end{align}
For instance $C_t$ could have up to a linear growth in time. In this case the decay of $\F(\nu_t)$ will no longer be in $\frac{1}{t}$ but only in $\frac{1}{\log(t)}$ \manote{This seems unlikely if we end up having convergence of $\nu_t$, but who nows.}. However, proving such results would be very involved and it is out of the scope of this study.
%One possible promising condition for \cref{eq:bounded_neg_sobolev} to hold would be if $\mu \ll \nu_0$ and if this property is preserved during the dynamics.


 








