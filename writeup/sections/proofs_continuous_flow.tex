\subsection{Proofs of \cref{sec:mmd_flow}}

\subsubsection{Proof of \cref{prop:mmd_flow}}

We firstly derive strong subdifferential of $\F$ associated with the 2-Wasserstein metric. In the case where $\F= \frac{1}{2} \|f_t\|^2_{\kH}$, by simple derivations we obtain:
\begin{equation}
 \nabla \frac{\partial\frac{1}{2} \|f_t\|^2_{\kH}}{\partial \rho_t}= \nabla \langle \frac{\partial f_t}{\partial \rho_t}, f_t \rangle_{\kH}= \nabla \langle \frac{\partial \E_{\rho_t}[k(Y,.)]}{\partial \rho_t}, f_t \rangle_{\kH}= \nabla \langle k(Y,.), f_t \rangle_{\kH}
\end{equation}
Then, by applying the reproducing property we have that:
\begin{equation}
\nabla \langle k(Y,.), f_t \rangle_{\kH}
= \nabla f_t(Y)
\end{equation}
where $\nabla f_t(Y)= \E_{X \sim \rho_t}[\nabla_{Y}k(X,Y)] -  \E_{X \sim \pi}[\nabla_{Y}k(X,Y)]$.

\subsubsection{Proof of \cref{prop:lambda_convexity} (Displacement convexity)}

We will firstly need the following lemma.

\begin{lemma}\label{lem:derivatives_witness}
	Let  $\mu$, $\nu_0$ and $\nu_1$ be three distributions in $\mathcal{P}_2(\X)$ and consider a displacement geodesic $(\rho_t)_{t\in[0,1]}$ between $\nu_0$ and $\nu_1$  defined by \cref{eq:displacement_geodesic} 
	and its corresponding velocity vector $(v_t)_{t\in [0,1]}$ as defined in \cref{eq:continuity_equation}. The following statements hold:
	\begin{enumerate}
		\item The first and second time derivatives of the witness function $f_{\mu,\rho_t}$ between $\mu$ and $\rho_t$ are well defined elements in $ \kH$ and are given by:
		\begin{align}\label{eq:derivatives_witness}
		\dot{f}_{\mu,\rho_t} = \int \nabla_1 k(x,.).v_t(x) \diff \rho_t(x); \qquad
		\ddot{f}_{\mu,\rho_t} = \int v_t(x)^T\nabla_1^2 k(x,.).v_t(x) \diff \rho_t(x)
		\end{align}
		where $ x \mapsto \nabla_1 k(x,z)$ and $x\mapsto \nabla_1^2 k(x,z)$ respectively denote the gradient and hessian of $x\mapsto k(x,z)$ for a fixed $z$ in $\X$.
		\item For all $g\in \kH$:
		\begin{align}\label{eq:inner_prod_deriative_witness}
		\langle g,\dot{f}_{\mu,\rho_t}\rangle_{\kH} = \int \nabla_1 g.v_t \diff \rho_t; \qquad
		\langle g,  \ddot{f}_{\mu,\rho_t}\rangle_{\kH} = \int v_t^T\nabla_1^2 g.v_t \diff \rho_t
		\end{align}
		\item The RKHS norms of $\dot{f}_{\mu,\rho_t}$ and $\ddot{f}_{\mu,\rho_t}$ satisfy:
		\begin{align}\label{eq:norm_derivative_witness}
		\Vert \dot{f}_{\mu,\rho_t}\Vert_{\kH}^2 = \langle v_t,C_{\rho_t} v_t \rangle_{L_2(\rho_t)}; \qquad  \Vert \ddot{f}_{\mu,\rho_t} \Vert\leq \lambda \Vert v_t \Vert^2_{L_2(\rho_t)}  
		\end{align}
		with $\lambda$ given by \cref{assump:bounded_fourth_oder} and $C_{\nu}$ defined in \cref{prop:lambda_convexity}. 
	\end{enumerate} 
\end{lemma}
\begin{proof}
	By definition of $\rho_{t}$:
	\[
	f_t(z)= \int k(x,z)\diff \mu(x) - \int k(s_t(x,y),z)\diff \pi(x,y)
	\]
	\manote{proof}
\end{proof}


\begin{proof}
To prove that $\nu\mapsto \F(\nu)$ is $\Lambda$-convex
we need to compute the second derivative $\ddot{\F}(\rho_{t})$
where $\rho_{t}$ is a displacement geodesic between two probability
distributions $\nu_{0}$ and $\nu_{1}$ as defined in \cref{eq:displacement_geodesic}. Such a minimizing geodesic always exists and can be written as $\rho_t = (s_t)_{\#}\pi$ with $s_t$ defined in \cref{eq:convex_combination} and $\pi$ is an optimal coupling between $\nu_0$ and $\nu_1$ (\cite{Santambrogio:2015}, Theorem 5.27). Moreover, we denote by $v_t$ the corresponding velocity vector as defined in \cref{eq:continuity_equation}. Recall from \cref{eq:mmd_norm_witness} that $\F(\rho_t) = \frac{1}{2} \Vert f_{\mu,\rho_t}\Vert^2_{\mathcal{H}}$, with $f_{\mu,\rho_t}$ defined in \cref{eq:witness_function}. To simplify notations we will write $f_t:= f_{\mu,\rho_t}$. We start by computing the first derivative of $ t\mapsto \F(\rho_t) $. By  \cref{lem:derivatives_witness},\cref{eq:derivatives_witness}, we know that $\dot{f}_t$ and $\ddot{f}_t $ are well defined elements of $\kH$ for any given $t\in [0,1]$, hence 
\[
 \dot{\F}(\rho_t) = \langle f_t, \dot{f_t}\rangle_{\kH};\qquad \ddot{\F}(\rho_t) = \Vert \dot{f_t}\Vert^2_{\kH} + \langle f_t, \ddot{f_t}\rangle_{\kH}.
 \]
While $\Vert \dot{f_t}\Vert^2_{\kH}$ is non-negative, $\langle f_t, \ddot{f_t}\rangle_{\kH}$ can in general be negative. We are only interested in quantifying how negative it can get, for this purpose we use Cauchy-Schwartz inequality which directly gives:
\[
\ddot{\F}(\rho_t)\geq  \Vert \dot{f}_t \Vert^2_{\kH} - \Vert f_t \Vert_{\kH}\Vert \ddot{f}_t\Vert_{\kH} 
\]

Finally by \cref{lem:derivatives_witness}, \cref{eq:norm_derivative_witness}, we can conclude that:
\[
	\ddot{\F}(\rho_t)\geq  \langle v_t,(C_{\rho_t} - \lambda \F(\rho_t)^{\frac{1}{2}}) v_t \rangle_{L_2(\rho_t)} 
\]
with $C_{\rho_t}$ given by \cref{eq:positive_operator_C} and $I$ is the identity operator in $L_2(\rho_t)$. Now we can introduce the function:
\begin{align}
	\Lambda(\nu,v) = \langle v ,( C_{\nu} -\lambda \F(\nu)^{\frac{1}{2}} I) v \rangle_{L_2(\nu)} 
\end{align}
which is defined for any pair $(\nu,v)$ with  $\nu\in \mathcal{P}_2(\X)$ and $v$ a square integrable vector field in $L_2(\nu)$. It is clear that $\Lambda(\nu,.)$  is a quadratic form on $L_2(\nu)$. Therefore, from \cref{def:lambda-convexity} of $\Lambda$ convexity, we conclude that $\F$ is $\Lambda$-convex.
\end{proof}


\subsubsection{Proof of \cref{cor:integral_lambda_convexity}}
\begin{proof}
	This is a direct consequence of the general identity (\cite{Villani:2009},
	Proposition 16.2). Indeed, for any continuous function $\phi$ on
	$[0,1]$ with second derivative $\ddot{\phi}$ that is bounded below
	in distribution sense the following identity holds:
	\[
	\phi(t)=(1-t)\phi(0)+t\phi(1)-\int_{0}^{1}\ddot{\phi}(s)G(s,t)ds
	\]
	Hence, one can choose $\phi(t)=\F(\rho_{t})$ therefore, \aknote{do we have second derivative bounded below for mmd?}
	it follows that:
	\[
	\F(\rho_{t})=(1-t)\F(\rho_{0})+t\F(\rho_{1})-\int_{0}^{1}\frac{d^{2}\F(\rho_{s})}{ds^{2}}G(s,t)ds
	\]
	Now using the inequality from \cref{prop:lambda_convexity}, \cref{eq:integral_lambda_convexity}
	follows directly. 
\end{proof}


\subsubsection{Proof of \cref{cor:loser_bound}}


\begin{proof}
	Recall the expression of $\Lambda(\rho_{s},v_{s}):$
	
	\[
	\Lambda(\rho_{s},v_{s})=\langle v_{t},(C_{\rho_{t}}-\lambda \F(\rho_{t})^\frac{1}{2} I)v_{t}\rangle_{L_{2}(\rho_{t})}\geq-\lambda \F(\rho_{t})^\frac{1}{2}\Vert v_{t}\Vert_{L_{2}(\rho_{t})}^{2}
	\]
	However, $\F(\rho_{t})^\frac{1}{2}\leq4C$ where $C=\sup_{x,y\in\mathcal{X}}\vert k(x,y)\vert$
	. Moreover, if $\rho_{t}$ is a constant speed geodesic then $\Vert v_{t}\Vert_{L_{2}(\rho_{t})}^{2}=W_{2}^{2}(\rho_{0},\rho_{1})$,
	hence: 
	\[
	-\int_{0}^{1}\Lambda(\rho_{s},v_{s})G(s,t)ds\leq\lambda 4CW_{2}^{2}(\rho_{0},\rho_{1})\int_{0}^{1}G(s,t)ds\leq2t(1-t)\lambda Cdiam(\mathcal{X})^{2}
	\]
	where $diam(\mathcal{X})$ is the diameter of $\mathcal{X}$. The rest of the proof follows by directly using \cref{cor:integral_lambda_convexity}
	and by setting $K=2\lambda Cdiam(\mathcal{X})$.
\end{proof}


\subsubsection{Proof of \cref{prop:almost_convex_optimization}}


\begin{proof}
	The proof is very similar to (\cite{Bottou:2017}, Theorem 6.3 and
	Theorem 6.9): \aknote{$\rho_1$ already taken}
	\begin{enumerate}
		\item First choose $\rho_{1}\in\mathcal{P}$ such that $\F(\rho_{1})<M-K$.
		For any $\rho_{0},\rho_{0}'\in L(\mathcal{P},M)$ there exist a displacement
		geodesic joining $\rho_{1}$ and $\rho_{0}$ without leaving $\mathcal{P}$,
		since $\mathcal{P}$ is by assumption discplacement convex. By \cref{cor:loser_bound}
		we have:
		\begin{align*}
		\F(\rho_{t}) & \leq(1-t)\F(\rho_{0})+t\F(\rho_{1})+t(1-t)K\\
		& \leq(1-t)M+t(M-K)+t(1-t)K\leq M-t^{2}K\leq M
		\end{align*}
		Hence $\rho_{t}\in L(\mathcal{P},M)$. The same can be done for a
		path joining $\rho_{0}'$ and $\rho_{1}$. Hence we can find a path
		in $L(\mathcal{P},M)$ joining $\rho_{0}$ and $\rho_{0}'$ , which
		means that the level set $L(\mathcal{P},M)$ is connected.
		\item Consider now $\rho_{1}\in L(\mathcal{P},M-K)$, note that such an
		element exists since $M>\inf_{\rho\in\mathcal{P}}\F(\rho)+K$.
		By convexity of $\mathcal{P}$ there exists a constant speed geodesic
		$\rho_{t}$ connecting $\rho_{0}$ and $\rho_{1}$. Since it is a
		constant speed curve then one has:
		\[
		W_{2}(\rho_{0},\rho_{t})\leq tW_{2}(\rho_{0},\rho_{1}).
		\]
		But we also have by \cref{cor:loser_bound}:
		\begin{align*}
		\F(\rho_{t}) & \leq(1-t)\F(\rho_{0})+t\F(\rho_{1})+t(1-t)K\\
		& \leq \F(\rho_{0})-t(\F(\rho_{0})-M+tK)\\
		& \leq \F(\rho_{0})-t(\F(\rho_{0})-M)
		\end{align*}
		Here we simply used the fact that $\rho_{1}\in L(\mathcal{P},M-K)$. 
	\end{enumerate}
\end{proof}







%


%
%
%
%By \cref{lem:derivatives_witness}, we have that $\dot{f_t}\in \kH$ and 
%
% it follows from \manote{some assumption to exchange orders}
%\[
%\frac{df_{t}}{dt}=\int(\nabla\phi(x)-x).\nabla k(\pi_{t}(x),.)\nu_{0}(x)dx
%\]
%hence:
%\[
%\frac{dMMD^{2}(\mu,\rho_{t})}{dt}=2\int(\nabla\phi(x)-x).\nabla f_{t}(\pi_{t}(x))\nu_{0}(x)dx
%\]
%Now the second derivative is given by:
%\begin{align*}
%\frac{d^{2}MMD^{2}(\mu,\rho_{t})}{dt^{2}}= & \int(\nabla\phi(x)-x).Hf_{t}(\pi_{t}(x))(\nabla\phi(x)-x)\nu_{0}(x)dx\\
% & +\int(\nabla\phi(x)-x).\nabla_{1}\nabla_{2}k(\pi_{t}(x),\pi_{t}(x'))(\nabla\phi(x')-x')\nu_{0}(x)\nu_{0}(x')dxdx'
%\end{align*}
%Here $\nabla_{1}\nabla_{2}k(x,x')$ is the matrix whose components
%are given by $\langle\partial_{i}k(x,.),\partial_{j}k(x,.)\rangle$
%for $1\leq i,j\leq d$, and $Hf_{t}$ is the hesssian of $f_{t}$
%and its components are also given by:
%\[
%(Hf_{t}(x))_{i,j}=\langle f_{t},\partial_{i}\partial_{j}k(x,.)\rangle.
%\]
%Denoting by $h(x):=\nabla\phi(x)-x$ it follows that:
%\begin{align*}
%\frac{d^{2}MMD^{2}(\mu,\rho_{t})}{dt^{2}}= & \langle f_{t},\int\sum_{i,j}h_{i}(x)h_{j}(x)\partial_{i}\partial_{j}k(\pi_{t}(x),.)\nu_{0}(x)dx\rangle\\
% & +\Vert\int\sum_{i}h_{i}(x)\partial_{i}k(\pi_{t}(x),.)\nu_{0}(x)dx\Vert^{2}
%\end{align*}
%Now we use Cauchy-Schwartz inequality for the first term to get:
%\begin{align*}
%\frac{d^{2}MMD^{2}(\mu,\rho_{t})}{dt^{2}}\geq & -\Vert f_{t}\Vert_{\kH}\Vert\int\sum_{i,j}h_{i}(x)h_{j}(x)\partial_{i}\partial_{j}k(\pi_{t}(x),.)\nu_{0}(x)dx\Vert_{\kH}\\
% & +\Vert\int\sum_{i}h_{i}(x)\partial_{i}k(\pi_{t}(x),.)\nu_{0}(x)dx\Vert^{2}.
%\end{align*}
%After applying a change of variables $x=\pi_{t}(y)$ one recovers the
%velocity vector $v_{t}$ instead of $h$: 
%\begin{align*}
%\frac{d^{2}MMD^{2}(\mu,\rho_{t})}{dt^{2}}\geq & -\Vert f_{t}\Vert_{\kH}\Vert\int\sum_{i,j}v_{t}^{i}(x)v_{t}^{j}(x)\partial_{i}\partial_{j}k(x,.)\rho_{t}(x)dx\Vert_{\kH}\\
% & +\Vert\int\sum_{i}v_{t}^{i}(x)\partial_{i}k(x,.)\rho_{t}(x)dx\Vert^{2}.
%\end{align*}
%
%One can further note that:
%\[
%\Vert\int\sum_{i,j}v_{t}^{i}(x)v_{t}^{j}(x)\partial_{i}\partial_{j}k(x,.)\rho_{t}(x)dx\Vert_{\kH}\leq\lambda\Vert v_{t}\Vert_{L_{2}(\rho_{t})}^{2}
%\]
%
%and that 
%\begin{align*}
%\Vert\int\sum_{i}v_{t}^{i}(x)\partial_{i}k(x,.)\rho_{t}(x)dx\Vert^{2} & =\int v_{t}(x)^{T}\int\nabla_{1}\nabla_{2}k(x,x')v_{t}(x')\rho_{t}(x')dx'dx.\\
% & =\langle v_{t},C_{\rho_{t}}v_{t}\rangle_{L_{2}(\rho_{t})}
%\end{align*}
%
%Hence we have shown that 
%\[
%\frac{d^{2}MMD^{2}(\mu,\rho_{t})}{dt^{2}}\geq\langle v_{t},(C_{\rho_{t}}-\lambda MMD(\mu,\rho_{t})I)v_{t}\rangle_{L_{2}(\rho_{t})}=\Lambda(\rho_{t},v_{t})
%\]