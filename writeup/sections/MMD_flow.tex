

\section{MMD gradient flow}\label{sec:mmd_flow}

\subsection{MMD as a free energy}

Interestingly, for a fixed target distribution $\mu$, it appears that the $MMD^2$ can be written
as a free energy \cref{eq:lyapunov}, by choosing the potential energy $V$ and interaction energy $W$ as follows:
\begin{align}
V(x)=-\int  k(x,x')\mu(x')\text{,} \quad
W(x,x')=\frac{1}{2}k(x,x')
\end{align}
Indeed, in this case we have $(1/2)MMD^2(\rho,\mu)=C+ \int V(x) \rho(x)dx + \int W(x,x')\rho(x)\rho(x')$, where $C=(1/2)\E_{\mu\otimes \mu}[k(x,x')]$. 
 Hence, we can consider a flow $(\rho_t)_{t>0}$ as described in \cref{sec:gradient_flows_functionals} and the loss functional defined as:
\begin{equation}\label{eq:mmd_functional}
\F(\rho_t)=\frac{1}{2}MMD^2(\rho_t, \mu)=\frac{1}{2}\|f_t\|^2_{\kH}
%&= \E_{\rho_t \otimes \rho_t}[k(X,X')]+\E_{\pi \otimes \pi}[k(Y,Y')] - 2\E_{\rho_t \otimes \pi}[k(X,Y)]
\end{equation} 
where $f_t(z)= \int k(.,z)\diff \mu - \int k(.,z)\diff \rho_t$ to alleviate notations. The following proposition gives the time dissipation of $\F$ as well as the associated gradient flow.

\begin{proposition}\label{prop:mmd_flow} The gradient flow associated to $\F$ and as the dissipation can be written:
\begin{equation}\label{eq:continuity_equation_mmd}
\frac{\partial \rho_t}{\partial t}= div(\rho_t  \nabla f_t) ,\quad  \quad \frac{d \F(\rho_t)}{dt}=-\E_{X \sim \rho_t}[\|\nabla f_t(X)\|^2]\quad
\end{equation}
 where $\nabla f_t$ is the gradient of the witness function $f_t$, defined by $\nabla f_t(z)= \int \nabla_{z}k(.,z) d\mu -  \int \nabla_{z}k(.,z) d\rho_t$. %Hence, the dissipation of $\F$ can be written:  
 %\begin{equation}
 %\frac{d \F(\rho_t)}{dt}=-\E_{X \sim \rho_t}[\|\nabla f_t(X)\|^2]\quad
 %\end{equation}
\end{proposition}
\begin{remark}
	If the functional $\F$ was the KL divergence and $\rho_t$ a weak solution of the Fokker-Planck equation \cref{eq:Fokker-Planck}, we would obtain the following dissipation (see \cite{wibisono2018sampling}):
	\begin{align}\label{eq:decreasing_mmd}
	\frac{d KL(\rho_t, \mu)}{dt}=-\E_{X \sim \rho_t}[\|\nabla log(\frac{\rho_t}{\mu}(X))\|^2]
	\end{align}
\end{remark}
A stochastic process whose distribution satisfies \cref{eq:continuity_equation_mmd} can thus be written (see \cref{sec:ito_stochastic} on Itô's formula):
\begin{equation}\label{eq:stochastic_process}
dX_t=-\nabla f_t(X_t) = - (\nabla V (X_t) + \nabla W * \rho_t(X_t))
\end{equation}
It can be interpreted as the position $X_t$ of a particle at time $t > 0$, following the velocity vector field $\nabla \frac{\partial{\F}}{\partial{\rho_t}}=\nabla f_t$.  Equation \eqref{eq:stochastic_process} is actually a McKean Vlasov model (see \cite{kac1956foundations}, \cite{mckean1966class}), a particular kind of SDE (Stochastic Differential Equation) driven by a Levy process:
\begin{align}\label{eq:theoretical_process}
&X_t=X_{0}+\int_{0}^t \sigma_{\mu}(X_s, \rho_s)ds \quad \text{for t in [0,T]}\\
&\forall s \in [0,T]\;,\quad \rho_s \text{ denotes the probability distribution of } X_s
\end{align}
with $\sigma_{\mu}(X_s, \rho_s)=-\nabla f_s(X_s)=\int \nabla_{X_s}k(.,X_s) d\rho_s -  \int \nabla_{X_s}k(.,X_s) d\mu$, and $X_0$ is distributed according to a given initial measure $\rho_0$. The non-linearity in the SDE \eqref{eq:theoretical_process} appears through the dependency of its coefficients on the law of the process. Suppose that $\nabla k$ is bounded and measurable on $\X$, and that there exists $L_k$ such that $\forall x,y \in \X$, $\|\nabla k(x,.)-\nabla k(y,.) \|_{\kH}\le L'_k \|x-y\|$. Hence, $\sigma$  Lipschitz continuous on $\X \times \mathcal{P}_2(\X)$ (endowed with the product of the canonical metric on $\X$ and $W_2$ on $\mathcal{P}_2(\X)$) and Equation~\eqref{eq:theoretical_process} admits a unique solution (see \cite{jourdain2007nonlinear}).
\aknote{What about the conditions for existence and uniqueness of the PDE\eqref{eq:continuity_equation_mmd}? does it relate to lambda convexity as santambrogio says? or Following Chizat Bach solutions exist for all t > 0 for appropriate initial $\mu_0$ that are compactly
	supported in $\X$?}
\begin{remark}
	Consider a family of particles whose density satisfy Equation\cref{eq:continuity_equation} for some free energy $\F$. Both KL and $MMD^2$ can be written as free energies \eqref{eq:lyapunov}, with a potential energy $V$ term which drive the particles to the target distribution $\mu$. While the entropy function $U$ in KL prevents the particle from "crashing" onto the mode of $\mu$, this role could be played by the interaction energy $W$ for $MMD^2$. Indeed, when $W$ is convex, this gives raise to a general aggregation behavior of the particles, while when it is not, the particles would push each other apart.\aknote{to check, ref malrieu?}
\end{remark}

 

%\section{Theoretical properties of the MMD flow}\label{sec:theory}



\subsection{Lambda displacement convexity of the MMD}

One important criterion to characterize the convergence of the gradient flow of a functional $\F$ is the notion of \textit{displacement convexity} of such a functional. Displacement convexity (see \cite{Villani:2004}, Definition 1). states that the functional evaluated at any distribution in a geodesic path between two distributions $\nu$ and $\nu'$ will be upper-bounded by a convex mixture of $\F(\nu)$ and $\F(\nu')$, as explained formally in the following definition.
\begin{definition}\label{def:displacement_convexity}
 Let $\mu$
and $\nu$ be two probabilities densities. There exists a $\mu-a.e.$
unique gradient of a convex function, denoted by $\nabla\phi$, such that $\nu$
is equal to $\nabla\phi_{\#}\mu$ and one can define \aknote{the displacement geodesic?} $\rho_{t}=((1-t)Id+t\nabla\phi)_{\#}\mu$
for $0\leq t\leq1$. We say that a functional $\nu\mapsto\mathcal{F}(\nu)$
is displacement convex if 
\[
t\mapsto\mathcal{F}(\rho_{t})
\]
 is convex for any $\mu$ and $\nu$. Moreover, we say that $\mathcal{F}$
is displacement convex in a neighborhood of $\mu$ if there exists a radius $r>0$
such that the above property holds for any $\nu$ with $W_{2}(\mu,\nu)\leq r$.
\end{definition}


This notion of convexity is to be related to the more widely used notion of convexity called \textit{mixture convexity}:
\begin{align}
	\F(t\nu +(1-t)\nu')\leq t\F(\nu)+(1-t)\F(\nu') \qquad t\in [0,1]
\end{align}
%Unlike mixture convexity, displacement convexity is compatible with the $W_2$ metric and is therefore the natural notion to use for characterizing convergence of gradient flows in the $W_2$ metric.
Although mixture convexity holds for $\F$ (see \cref{lem:mixture_convexity}), this property is less critical for characterizing convergence of gradient flows in the $W_2$ metric. On the other hand, displacement convexity is compatible with the $W_2$ metric \cite{Bottou:2017} and is therefore the natural notion to use in our setting. Unfortunately, $\F$ fails to be displacement convex in general. Instead we will show that $\F$ satisfies some weaker notion of convexity called $\Lambda$-displacement convexity:
%
\begin{definition}\label{def:lambda-convexity}
($\Lambda$-convexity \cite{Villani:2009} Definition 16.4). Let $(\mu,v)\mapsto\Lambda(\mu,v)$
be a function that defines for each probability distribution $\mu$
a quadratic form on the set of square integrable vectors valued functions
$v$ , i.e: $v\in L_{2}(\mathbb{R}^{d},\mathbb{R}^{d},\mu)$ . We
further assume that:
\[
\inf_{\mu,v}\frac{\Lambda(\mu,v)}{\Vert v\Vert_{L_{2}(\mu)}^{2}}>-\infty.
\]
We say that a functional $\mu\mapsto\mathcal{F}(\mu)$ is $\Lambda$-convex
if for any $\mu$ and $\nu$ and a minimizing geodesic $\text{\ensuremath{\rho_{t}}}$
between $\mu$ and $\nu$ with velocity vector field $v_{t}$, i.e:
$\partial_{t}\rho_{t}+div(\rho_{t}v_{t})=0;\rho_{0}=\mu;\rho_{1}=\nu$
the following holds:
\begin{equation*}
\frac{d^{2}\mathcal{F}(\rho_{t})}{dt^{2}}\geq\Lambda(\rho_{t},v_{t})\qquad\forall t\in[0,1].
\end{equation*}
\end{definition}

To show the $\Lambda$-convexity of the functional defined in \cref{eq:MMD_functional} we first make the following assumptions on the kernel:
\begin{assumplist} 
\item \label{assump:bounded_trace} $ \vert \sum_{1\leq i\leq d} \partial_i\partial_ik(x,x) \vert\leq \frac{L}{3}  $ for all $x\in \mathbb{R}^d$.
\item \label{assump:bounded_hessian} $\Vert H_xk(x,y) \Vert_{op} \leq \frac{L}{3}$ for all $x,y\in \mathbb{R}^d$, where $H_xk(x,y)$ is the hessian of $x\mapsto k(x,y)$ and $\Vert.\Vert_{op}$ is the operator norm.
\item \label{assump:bounded_fourth_oder} $\Vert Dk(x,y) \Vert\leq \lambda  $ for all $x,y\in \mathbb{R}$, where $Dk(x,y)$ is an $\mathbb{R}^{d^2}\times \mathbb{R}^{d^2}$ matrix with entries given by $\partial_{x_{i}}\partial_{x_{j}}\partial_{x'_{i}}\partial_{x_{j}'}k(x,x')$.
\end{assumplist}\aknote{do we have an order of magnitude for lambda? or just we put a remark to say it's satisfied by the gaussian kernel}
The next proposition states that the functional defined in \cref{eq:MMD_functional} is $\Lambda$-displacement convex and provide and explicit expression for the functional $\Lambda$.

\begin{proposition}
\label{prop:lambda_convexity} Suppose \cref{assump:bounded_fourth_oder} is satisfied for some $\lambda \in \R^+$. The functional $\nu\mapsto \F(\nu)$ is $\text{\ensuremath{\Lambda}}$-convex
with $\Lambda$ given by:
\begin{equation}
\Lambda(\rho,v)=\langle v,(C_{\rho}-\lambda \F(\rho)^{\frac{1}{2}}I)v\rangle_{L_{2}(\rho)}\label{eq:Lambda}
\end{equation}
where $C_{\rho}$ is the (positive) operator defined by:
\begin{align}\label{eq:positive_operator_C}
	(C_{\rho}v)(x)=\int\nabla_{x}\nabla_{x'}k(x,x')v(x')d\rho(x')
\end{align}
\end{proposition}
%
%
Consider the geodesic \aknote{path/geodesic/curve?}$\rho_{t}=((1-t)Id+t\nabla\phi)_{\#}\mu$ of \cref{def:displacement_convexity}. It is worth noting that $\rho_{0}=\mu$ and at time $t=0$ we have
that $\F(\rho_{0})=0$, hence we get:
\[
\frac{d^{2}\F(\rho_{t})}{dt^{2}}\vert_{t=0}=\langle v_{t},C_{\rho_{t}}v_{t}\rangle_{L_{2}(\rho_{t})}\geq0.
\]
This shows that $\nu\mapsto \F(\nu)$ has a non-negative
hessian at $\mu$ which is not surprising since $\mu$ is the global
minimum of this functional.
\begin{corollary}\label{cor:integral_lambda_convexity}
For any geodesic $\rho_{t}$ between two probability distributions
$\rho_{0}$ and $\rho_{1}$ the following holds:
\begin{equation}
\F(\rho_{t})\leq(1-t)\F(\rho_{0})+t\F(\rho_{1})-\int_{0}^{1}\Lambda(\rho_{s},v_{s})G(s,t)ds\label{eq:integral_lambda_convexity}
\end{equation}
where $\Lambda$ is given by \cref{eq:Lambda} and $G$ is given
by:
\[
G(s,t)=\begin{cases}
s(1-t) & s\leq t\\
t(1-s) & s\geq t
\end{cases}
\]
\end{corollary}
%

\begin{corollary}
\label{cor:loser_bound}Assume the distributions are supported on
$\mathcal{X}$ and the kernel is bounded, i.e: $\sup_{x,y\in\mathcal{X}}\vert k(x,y)\vert<\infty$.
Then the following holds:
\begin{equation}
\F(\rho_{t})\leq(1-t)\F(\rho_{0})+t\F(\rho_{1})+t(1-t)K
\end{equation}
where $K$ is a constant depending on $\X$ and the kernel $k$ in $\F$.
\end{corollary}
%
%
\cref{cor:loser_bound}, is a loser bound and does not account for the local
convexity of the MMD. However, it allows to state the following result,
which is inspired from (\cite{Bottou:2017}, Theorem 6.3) but generalizes
it to the case of 'almost convex' functionals.
\begin{proposition}
\label{prop:almost_convex_optimization}
(Almost convex optimization). Let $\mathcal{P}$ be a closed subset
of $\mathcal{P}(\mathcal{X})$ which is displacement convex\aknote{weird for a set to be displacement convex? it was for functionals}. Then
for all $M>\inf_{\rho\in\mathcal{P}}\F(\rho)+K$, the following
holds:
\end{proposition}
\begin{enumerate}
\item The level set $L(\mathcal{P},M)=\{\rho\in\mathcal{P}:\F(\rho)\leq M\}$
is connected
\item For all $\rho_{0}\in\mathcal{P}$ such that $\F(\rho_0)>M$
and all $\epsilon>0$, there exists $\rho\in\mathcal{P}$ such that
$W_{2}(\rho,\rho_{0})=\mathcal{O}(\epsilon)$ and
\[
\F(\rho)\leq \F(\rho_{0})-\epsilon(\F(\rho_{0})-M).
\]
\end{enumerate}
%
\begin{remark}
The result in \Cref{prop:almost_convex_optimization} means that it is possible to optimize the cost function $\rho\mapsto \F(\rho)$
on $\mathcal{P}$ as long as the barrier $\inf_{\rho\in\mathcal{P}}\F(\rho)+K$
is not reached. We provide now a simple proof of this result.
\end{remark}


\begin{remark}
	A possible direction would be to directly leverage the tighter inequality in \cref{eq:integral_lambda_convexity} to get a better description of the loss landscape.
\end{remark}








\subsection{Lojasiewicz type inequality - Convergence of the continuous flow}

Here we would like to derive an inequality between the time derivative of the Lyapounov functional $\mathcal{F}$ along its gradient flow $t\mapsto \nu_t$. For this purpose we first introduce the weighted negative Sobolev distance \manote{cite villani and peyre and Mroueh}:
\begin{align}\label{eq:neg_sobolev}
	\Vert \nu - \mu \Vert_{\dot{H}^{-1}(\nu)} = \sup_{\substack{ f\in W_0^{1,2}(\nu) \\ \nu(\Vert \nabla f \Vert^2) \leq 1 }} \vert \nu(f)-\mu(f)\vert 
\end{align}
Where $W_0^{1,2}(\nu)$ is the space $1$ order Sobolev functions with functions vanishing at the boundary of the domain.
The distance defined in \cref{eq:neg_sobolev} plays a fundamental role in dynamic optimal transport as it linearizes the $W_2$ distance when $\mu$ is arbitrarily close to $\nu$. It can also be seen as the minimum kinetic energy needed to advect the mass $\nu$ to $\mu$. However, this quantity might be infinite \manote{say exactly when it is finite} and one of the key problems would be to control its value during the evolution of the flow. More precisely we will rely on the following statement:
\begin{align}\label{eq:bounded_neg_sobolev}
	\Vert \nu_t  - \mu \Vert_{\dot{H}^{-1}(\nu_t)} \leq C \qquad \forall t\geq 0.
\end{align} 
where $\nu_t$ is defined by the gradient flow and $\mu$ is the target distribution. When \cref{eq:bounded_neg_sobolev}  holds, we have the following proposition:
\begin{proposition}\label{prop:PL_type_inequality}
	When \cref{eq:bounded_neg_sobolev} holds, the following inequality is then satisfied at all times:
	\begin{align}\label{eq:PL_type_inequality}
		\Vert \nabla f_t \Vert_{L_2(\nu_t)} \geq \frac{1}{C} \Vert f_t \Vert^2_{\mathcal{H}} \qquad \forall t\geq 0.
	\end{align}
\end{proposition}
\begin{proof}
	Indeed, this follows simply from the definition of the negative Sobolev distance: Consider $g = \Vert \nabla f_t\Vert^{-1}_{L_2(\nu_t)} f_t$, then $g\in W_0^{1,2}(\nu)$ \manote{this suggests an assumption on the kernel so that all those function satisfy a boundary condition} and $\Vert \nabla g \Vert_{L_2(\nu_t)}\leq 1$. Therefore, we directly have:
	\begin{align}
		\Vert \nu_t - \mu\Vert_{\dot{H}^{-1}(\nu_t)}\geq \vert \nu_t(g) - \mu(g)  \vert.
	\end{align}
Now, recall the definition of $g$, which implies that
\[
\vert \nu_t(g) - \mu(g)  \vert = \Vert \nabla f_t\Vert^{-1}_{L_2(\nu_t)} \vert \nu_t(f_t)-\mu(f_t)\vert.
\]
But since $f_t$  is exactly the witness functions between $\nu_t$ and $\mu$, it follows that $\nu_t(f_t)-\mu(f_t) = \Vert f_t\Vert^2_{\kH}$.
Using \cref{eq:bounded_neg_sobolev}, we get the desired inequality.
\end{proof}

Now we will use the inequality in \cref{prop:PL_type_inequality} to prove a convergence result towards the global optimum $\mu$. This is provided in \cref{prop:convergence}.

\begin{proposition}\label{prop:convergence}
	If \cref{eq:bounded_neg_sobolev} is satisfied for all times then $t\mapsto \mathcal{F}(\nu_t)$ converges to $0$ with a rate of convergence given by:
	\begin{align}
		\mathcal{F}(\nu_t)\leq \frac{1}{\mathcal{F}(\nu_0)^{-1} + \frac{4t}{C}}
	\end{align}
\end{proposition}
\begin{proof}
	The proof is a simple consequence of \cref{prop:mmd_flow,eq:bounded_neg_sobolev}. Indeed, by \cref{prop:mmd_flow} we have that 
	\begin{align}
		\dot{\F}(\nu_t) = - \Vert \nabla f_t \Vert^2_{L_2(\nu_t)} 	
	\end{align}
	Using \cref{eq:PL_type_inequality}, we directly get that:
	\begin{align}
		\dot{\F}(\nu_t) \leq  -\frac{4}{C}\F(\nu_t)^2
	\end{align}
It is clear that if $\mathcal{F}(\nu_0)>0$ then $\F(\nu_t)>0$ at all times by uniqueness of the solution. Hence, one can divide by $\F(\nu_t)^2$ and integrate the inequality from $0$ to some time $t$. The desired inequality is obtained by simple calculations.
\end{proof}

All the difficulty is to see now when \cref{eq:bounded_neg_sobolev} holds. One possible strategy would be to start from initial $\nu_0$ such that $\Vert \nu_0  - \mu \Vert_{\dot{H}^{-1}(\nu_0)} \leq C $  for some finite positive value $C$ and then show that this property is preserved during the dynamics. It is also possible to have a time depended constant $C_t$ as long as its growth is such that:
\begin{align}
	\lim_{t\rightarrow +\infty} \int_0^t C_s^{-1}\diff s = +\infty
\end{align}
For instance $C_t$ could have up to a linear growth in time. In this case the decay of $\F(\nu_t)$ will no longer be in $\frac{1}{t}$ but only in $\frac{1}{\log(t)}$ \manote{This seems unlikely if we end up having convergence of $\nu_t$, but who nows.}.
One possible promising condition for \cref{eq:bounded_neg_sobolev} to hold would be if $\mu \ll \nu_0$ and if this property is preserved during the dynamics.


 












\subsection{Noisy MMD flow}\label{sec:noisy_flow}

Although the Wasserstein flow of the MMD decreases the MMD in time, it can very well remain stuck in local minima. This can happen when the negative sobolev norm (see section \cref{sec:Lojasiewicz_inequality} is not finite at all times. One way to see this, at least formally, is by looking at the equilibrium condition for \cref{eq:time_evolution_mmd}.
%\asnote{I think that the same problem happens with the dynamics of SVGD. Because KSD = 0 doesn't imply p = q unless absolute continuity + other requirements} 
Indeed $\F(\nu_t)$ is a non-negative decreasing function of time, it must therefore converge to some limit, which implies in turn that its time derivative would also converge to $0$. Assuming that $\nu_t$ also converged to some limit distribution $\nu^{*}$
\footnote{There are cases when $\nu_t$ doesn't converge to any distribution. This would happen is the sequence $(\nu_t)_{t\geq 0}$ is not tight.} 
one can show that under simple regularity conditions\aknote{which ones?} that the equilibrium condition
\begin{align}\label{eq:equilibrium_condition}
	\int \Vert \nabla f_{\mu,\nu^{*}}(x)\Vert^2 \diff \nu^{*}(x) =0  
\end{align}
must hold. If $\nu^*$ turns out to have a positive density, this would imply that $f_{\mu,\nu^{*}}(x)$ is constant everywhere. This in turn would mean that $f_{\mu,\nu^{*}}=0$ when the RKHS doesn't contain constant functions\footnote{This is the case for the gaussian kernel for instance}. Hence, $\nu^*$ would be a global optimum since $\F(\nu^{*})=0$. However, the limit distribution $\nu^*$  might be very singular, it could even be a dirac distribution. \manote{here a figure would be nice}  This suggests that the gradient flow could converge to a suboptimal solution $\nu^*$ for which \cref{eq:equilibrium_condition} is true. 
Since \cref{eq:equilibrium_condition} seems to be the main obstruction to reach global optimality, we propose an approximate gradient descent algorithm, which aims at avoiding local minima by injecting noise into the gradient at each iteration $n$:\aknote{abrupt}  
\begin{align}\label{eq:discretized_noisy_flow}
	X_{n+1} = X_{n} -\gamma \nabla f_{\mu,\nu_n}(X_n+ \beta_n U_n) \qquad n\geq 0
\end{align}
where $U_n \sim \mathcal{N}(0,1)$ and $\beta_n$ is the noise level. Unlike in \cref{eq:euler_scheme}, here the sample is blurred first before evaluating the gradient.
Intuitively, if $\nu_n$ approaches a local optimum $\nu^{*}$, $ \nabla f_{\mu,\nu_n}$ would be small on the support of $\nu_n$ but it might be much larger outside of it, hence evaluating $\nabla f_{\mu,\nu_n}$ outside the support of $\nu_n$ might help escaping the local minimum. We show in \manote{add this in the appendix} that \cref{eq:discretized_noisy_flow} is associated to an augmented McKean-Vlasov process that is different from adding a diffusion term to \cref{eq:continuity_mmd}. Indeed, in the later case, the update equation would be:
\begin{align}\label{eq:diffusion}
	X_{n+1} = X_{n} -\gamma \nabla f_{\mu,\nu_n}(X_n)+ \beta_n U_n \qquad n\geq 0.
\end{align}
%to construct, at least formally, a modified gradient flow for which the optimality condition would guarantee reaching the global optimum.
%Ideally, we would like to obtain an optimality condition of the form
%\begin{align}\label{eq:soothed_equilibrium_condition}
%	\int \Vert \nabla f_{\mu,\nu^{*}}(x)\Vert^2 \diff (\nu^{*}\star g)(x) =0  
%\end{align}
%where $\nu^{*}\star g$ means the convolution of $\nu^*$ with a gaussian distribution $g$. The smoothing effect of convolution directly implies that $\nu^{*}\star g$ has a positive density, which falls back in the scenario where the $\nu^*$ must a global optimum.
%We consider, at least formally, the following modified equation for $\nu_t$:
%\begin{align}\label{eq:smoothed_continuity_equation_mmd}
%	\partial_t \nu_t = div((\nu_t \star g) \nabla f_{\mu,\nu_t} )
%\end{align}
%This suggests a particle equation which would be given by:
%\begin{align}\label{eq:noisy_particles}
%	\dot{X}_t = -\nabla f_{\mu,\nu_t}( X_t + W_t  )
%\end{align}
%where $(W_t)$ is a brownian motion. Furthermore, $\F(\nu_t)$ satisfies
%\begin{align}\label{eq:smoothed_decreasing_mmd}
%	\dot{\F}(\nu_t) = -\int \Vert \nabla f_{\mu,\nu_t}(x)\Vert^2 \diff (\nu_t\star g)(x)
%\end{align}
%The existence and uniqueness of a solution to \cref{eq:smoothed_continuity_equation_mmd} for a general $g$ remains an open question to our knowledge. However, we find it useful here to state \cref{eq:smoothed_continuity_equation_mmd,eq:noisy_particles,eq:smoothed_decreasing_mmd} which are the modified analogs of \manote{ref to the analogs}.
 %\cref{eq:diffusion} 
which corresponds to regularizing $\F$ using an entropic term as in \cite{mei2018mean,Simsekli:2018}. Our proposal \cref{eq:discretized_noisy_flow} is also different from \cite{craig2016blob,carrillo2019blob} where $\F$ is regularized by convolving the interaction potential $W$ in \cref{eq:potentials}. However, the optimal solution of a regularized version of the functional $\F$ will be generally different from the non-regularized one, which is not desirable in our setting. This is not the case for \cref{eq:discretized_noisy_flow} where the global optimum of $\F$ is a fixed point. \aknote{really?} 
 %As shown in \manote{add this in the appendix}, \cref{eq:discretized_noisy_flow} is associated to an augmented continuous-time dynamics  which decreases $\F$ under a condition on the noise level $\beta_k$:
In fact \cref{eq:discretized_noisy_flow} is  closely related to the \textit{continuation methods} \cite{Gulcehre:2016a,Gulcehre:2016,Chaudhari:2017}  and \textit{graduated optimization} \cite{Hazan:2015} used for non-convex optimization in Euclidian spaces. Indeed given a non-convex cost function $f$, the graduated descent would lead to updates of the form: $X_{n+1} = X_n - \gamma \nabla f(X_n+\beta U_n )$. The main difference with \cref{eq:discretized_noisy_flow} is the dependence of $f$ on $\nu_n$ which is inherently due to functional optimization.
We show in the following proposition, whose proof is provided in \cref{eq:proof_decreasing_noisy_loss}, that \cref{eq:discretized_noisy_flow} decreases the loss functional at every iteration provided that the level of the noise is well controlled.
\begin{proposition}\label{prop:decreasing_loss_iterations}
	Let $(\nu_n)_{n\geq 0}$ be the sequence of distributions defined by \cref{eq:discretized_noisy_flow} with an initial condition $\nu_0$. Under \cref{assump:bounded_hessian}, and for a choice of $\beta_n$ such that:
	\begin{align}\label{eq:control_level_noise}
		8L^2\beta_n^2 \F(\nu_n) \leq \int \Vert \nabla f_{\mu,\nu_n}(x+\beta_n u) \Vert^2 g(u) \diff \nu_n(x)\diff u   
	\end{align}
	 the following inequality holds:
	\begin{align}\label{eq:decreasing_loss_iterations}
		\F(\nu_{n+1}) - \F(\nu_n  ) \leq -\frac{\gamma}{2}(1-\gamma L)\int \Vert \nabla f_{\mu,\nu_n}(x+\beta_n u) \Vert^2 g(u) \diff\nu_n(x) \diff u
	\end{align}
	Here $L$ is given in \cref{assump:bounded_hessian} and depends only on the choice of the kernel, and $g$ is the density of the standard gaussian distribution.
\end{proposition}
%A proof of \cref{prop:decreasing_loss_iterations} is provided in \cref{eq:proof_decreasing_noisy_loss}.

\begin{remark}
	  %This allows the algorithm to use non-local information on the loss landscape by probing the gradient in regions outside of the support of $\nu_k$. Thus this algorithm could potentially escape local optima. 
	At each iteration, the level of the noise needs to be adjusted such that the gradient is not too much blurred. This ensures that each step would decrease the loss functional. However, $\beta_n$ doesn't need to decrease at each iteration, it could increase adaptively whenever needed, i.e. when  the sequence gets closer to a local optimum, it is helpful to increase the level of the noise to probe the gradient in regions where its value is not flat.
	Finally, \cref{eq:decreasing_loss_iterations} is always satisfied for $\beta_n = 0$ where we recover the noise-free discretized flow. However, the interesting cases are when $\beta_n>0$.
	%The second crucial point, is the dependence of the level of the noise on the value of the loss functional itself in \cref{eq:control_level_noise}. This allows some tolerance for high levels of noise when the loss functional is already small. In fact this precise condition provides a Lojasiewicz type inequality for free, which will then be used in  to provide convergence rates in  \cref{sec:Lojasiewicz_inequality}.
 \end{remark}
The natural question is whether \cref{eq:discretized_noisy_flow} converges towards to global optimum of $\F$. The answer will depend on how much noise is allowed to be injected while still decreasing $\F$. The higher the $\beta_n$ is, the faster it will converge. This is made more precise in \cref{thm:convergence_noisy_gradient}: 
 \begin{theorem}\label{thm:convergence_noisy_gradient}
 Assume that $\sum_{i=0}^n \beta_i^2 \rightarrow \infty $ and \cref{eq:control_level_noise} is satisfied for all $n$ then:
 \begin{align}
 	\F(\nu_n)\leq \F(\nu_0) e^{-4L^2\gamma(1-\gamma L)\sum_{i=0}^n \beta^2_i}
 \end{align}
 \end{theorem}
 A proof of \cref{thm:convergence_noisy_gradient} is provided in \manote{add proof} and relies on \cref{eq:control_level_noise} to get a Lojasiewicz inequality which then controls the decay of $\F(\nu_n)$. A particular case when \cref{thm:convergence_noisy_gradient} holds is when $\beta_n$ is guaranteed to be always greater than a minimal value $\beta^*>0$ while still having \cref{eq:control_level_noise}. In this case, one recovers linear convergence rates for $\F(\nu_n)$.
In \label{subsec:euler_maruyama} we provide a practical algorithm which performs noisy gradient updates as in \cref{eq:discretized_noisy_flow} and provide guarantees for this algorithm.
 
 
 



%\subsection{MMD flows in the literature}

%\begin{remark}
%	We point out here that algorithm~\cref{eq:sample_based_process} is different from the descent proposed by \cite{mroueh2018regularized}. 
%\end{remark}

%\begin{remark}
%	Birth-Death Dynamics to improve convergence (see \cite{rotskoff2019global}).
%\end{remark}
