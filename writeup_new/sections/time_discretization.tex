

%\subsection{Convergence of the time-discretized flow}
%
%The time discretized flow can be written:\aknote{add pushforward $\#$ definitions}
%\begin{align}\label{eq:discretized_flow}
%\nu_{m+1} = (I -\gamma \phi_m)_{\#}\nu_m
%\end{align}
%where $\gamma$ is some fixed step-size and $X \mapsto \phi_m(X):=\nabla f_{\mu, \nu_m}(X)$ is the gradient of the witness function between $\mu$ and $\nu_m$.% It is easy to see that the particle version of \cref{eq:discretized_flow} is given by:
%%\begin{align}
%%X_{n+1} = X_n - \gamma \phi_n(X_n) \quad\forall n\in \mathbb{N}.
%%\end{align} 
%\cref{prop:almost_convex_optimization} guarantees the existence of a direction of descent that minimizes the functional $\F$ provided that the starting point $\rho_1$ has a potential greater than the barrier $K$.%, i.e:
%%\begin{align}\label{eq:barrier_condition}
%%	\F(\rho_1)> \inf_{\rho\in \mathcal{P}} \F(\rho) + K
%%\end{align}
%One natural question to ask is whether the  discretized gradient flow algorithm provides such way to reach the barrier $K$ and at what speed this happens. This subsection will answer that question. Firstly, we state few propositions that will lead us to the final result.
%
%
%%\begin{proposition}\label{prop:decreasing_functional}
%%	Under \cref{assump:bounded_trace,assump:bounded_hessian}, the following inequality holds:
%%	\begin{align*}
%%	\F(\nu_{n+1})-\F(\nu_n)\leq -\gamma (1-\frac{\gamma}{2}L )\int \Vert \phi_n(X)\Vert^2 d\nu_n
%%	\end{align*}
%%\end{proposition}
%
%\begin{proposition}\label{prop:evi}
%	Consider the sequence of distributions $\nu_m$ obtained from \cref{eq:discretized_flow}. If $\gamma \leq 1/L$, then
%	\begin{align}
%2\gamma(\F(\nu_{m+1})-\F(\mu))
%\leq 
%W_2^2(\nu_m,\mu)-W_2^2(\nu_{m+1},\mu)-2\gamma K(\rho^m).
%\label{eq:evi}
%\end{align}
%where $(\rho^m_t)_{0\leq t \leq 1}$ is a constant-speed geodesic from $\nu_n$ to $\mu$ and $K(\rho^m):=\int_0^1 \Lambda(\rho^m_s,\dot{\rho^m}_s)(1-s)ds$.
%\end{proposition}
%
%\begin{theorem}\label{th:rates_mmd}
%	Consider the sequence of distributions $\nu_n$ obtained from \cref{eq:discretized_flow}. If $\gamma \leq 1/L$, then
%	%\begin{align}
%%\F(\bar{\nu}_{n})-\F(\mu)\leq  \frac{W_2^2(\nu_0,\mu)}{2 \gamma n} -\bar{K}
%%\end{align}
%%where $\bar{\nu}=\frac{1}{N}\sum_{n=1}^N \nu_n$. Moreover, 
%\begin{align}
%\F(\nu_m)-\F(\mu)\leq  \frac{W_2^2(\nu_0,\mu)}{2 \gamma m} -\bar{K}.
%\end{align}
%\end{theorem}

