

\subsection{Additional mathematical background}

\subsubsection{MMD in Reproducing Kernel Hilbert Spaces (RKHS)}\label{sec:rkhs}

We recall here fundamental definitions and properties of reproducing kernel Hilbert spaces (RKHS) (see \cite{smola1998learning}) and Maximum Mean Discrepancies (MMD). The key aspect of a RKHS is
the reproducing property: for all $f \in \kH, f(x) = \langle f, k(x, .)\rangle_{\kH}$. 
Suppose that $k(.,.))$ is measurable and that $\E_x[k(x,x)]<\infty$.
Given $\mathcal{P}(\X)$ the set of probability measures defined on $\X$, $k$
is said to be characteristic if:
\begin{equation}
\mu \mapsto \int k(.,x) d\mu(x)
\end{equation}
is injective, i.e. $\mu$ is mapped to a unique element in $\kH$ called its mean embedding.  Suppose additionally that $k(.,.)$ is continuous, $\X$ is compact, and that $\kH$ is dense in $C(\X)$ the space of continuous bounded functions on $\X$. Under these conditions, the MMD is a metric (\cite{gretton2012kernel}, Theorem 5).

%\subsubsection{Optimal transport}

\subsection{Background on optimal transport}

In this section we recall how to endow the space of probability measures $\mathcal{P}(\X)$ on $\X$ a compact, convex subset of $\R^d$ with a distance (e.g, optimal transport distances), and then deal with gradient flows of suitable functionals on such a metric space. The reader may refer to %\cite{santambrogio2017euclidean} for a clear review on the subject. For a given distributions $\nu\in\mathcal{P}(\X)$ and an integrable function $f$ under $\nu$, the expectation of $f$ under $\nu$ will be written either as $\nu(f)$ or $\int f \diff\nu$ depending on the context. 

\subsubsection{$2$-Wasserstein geometry}\label{subsec:wasserstein_flow}

Let $T: \X \rightarrow \X$ be a measurable map, and $\rho \in \mathcal{P}(\X)$. The push-forward measure $T_{\#}\rho$
is characterized by:
\begin{align}
%	&\quad T_{\#}\rho(A) = \rho(T^{-1}(A)) \text{ for every measurable set A,}\\
%\text{or}&
 \int_{y \in \X} \phi(y) d(T_{\#}\rho)(y) =\int_{x \in \X}\phi(T(x)) d\rho(x) \text{ for every measurable function $\phi$.}
\end{align}
Let $\mathcal{P}_2(\X)$ the set of probability distributions on $\X$ with finite second moment. For two given probability distributions $\nu$ and $\mu$ in $\mathcal{P}_2(\X)$ we denote by $\Pi(\nu,\mu)$ the set of possible couplings between $\nu$ and $\mu$. In other words $\Pi(\nu,\mu)$ contains all possible distributions $\pi$ on $\X\times \X$ such that if $(X,Y) \sim \pi $ then $X \sim \nu $ and $Y\sim \mu$. The $2$-Wasserstein distance on $\mathcal{P}_2(\X)$ is defined by means of optimal coupling between $\nu$ and $\mu$ in the following way:
\begin{align}\label{eq:wasserstein_2}
W_2^2(\nu,\mu) := \inf_{\pi\in\Pi(\nu,\mu)} \int \Vert x - y\Vert^2 d\pi(x,y) \qquad \forall \nu, \mu\in \mathcal{P}_2(\X)
\end{align}
It is a well established fact that such optimal coupling $\pi^*$ exists. Moreover, it can be used to define a path $(\rho_t)_{t\in [0,1]}$ between $\nu$ and $\mu$ in $\mathcal{P}_2(\X)$. \aknote{Maybe we can defer the def of pushforward measures and equation of $s_t$ to the Appendix?}For a given time $t$ in $[0,1]$ and given a sample $(x,y)$ from $\pi^{*}$, it possible to construct a sample $z_t$ from $\rho_t$ by taking the convex combination of $x$ and $y$: $z_t = s_t(x,y)$ where $s_t$ is given by \cref{eq:convex_combination}
\begin{equation}\label{eq:convex_combination}
s_t(x,y) = (1-t)x+ty \qquad \forall x,y\in \X, \; \forall t\in [0,1].
\end{equation}
The function $s_t$ is well defined since $\X$ is a convex set. More formally, $\rho_t$ can be written as the projection or push-forward of the optimal coupling $\pi^{*}$ by $s_t$:  
\begin{equation}\label{eq:displacement_geodesic}
\rho_t = (s_t)_{\#}\pi^{*}
\end{equation}
It is easy to see that \cref{eq:displacement_geodesic} satisfies the following boundary conditions:
\begin{align}\label{eq:boundary_conditions}
\rho_0 = \nu \qquad \rho_1 = \mu.
\end{align}
Paths of the form of \cref{eq:displacement_geodesic} are called \textit{displacement geodesics}. They can be seen as the shortest paths from $\nu$ to $\mu$ in terms of mass transport (\cite{Santambrogio:2015} Theorem 5.27). It can be shown that there exists a \textit{velocity vector field} $(t,x)\mapsto v_t(x)$ with values in $\R^d$ such that $\rho_t$ satisfies the continuity equation:
\begin{equation}\label{eq:continuity_equation}
\partial_t \rho_t + div(\rho_t v_t ) = 0 \qquad \forall t\in[0,1].
\end{equation}
Equation \cref{eq:continuity_equation} is well defined in distribution sense even when $\rho_t$ doesn't have a density. $v_t$ can be interpreted as a tangent vector to the curve $(\rho_t)_{t\in[0,1]}$ at time $t$ so that the length $l(\rho_t)$ of the curve $\rho_t$ would be given by:
\begin{equation}
l(\rho)^2 = \int_0^1 \Vert v_t \Vert^2_{L_2(\rho_t)} \diff t \quad \text{ where } \quad 
\Vert v_t \Vert^2_{L_2(\rho_t)} =  \int \Vert v_t(x) \Vert^2 \diff \rho_t(x)
\end{equation}
%\aknote{add constant speed geodesics}
This perspective allows to provide a dynamical interpretation of the $W_2$ as the length  of the shortest path from $\nu$ to $\mu$ and is summarized by the celebrated Benamou-Brenier formula (\cite{Santambrogio:2015}, Theorem\aknote{check} 5.28):
\begin{align}\label{eq:benamou-brenier-formula}
W_2(\nu,\mu) = \inf_{(\rho,v)} l(\rho)
\end{align}
where the infimum is taken  over all couples  $\rho$ and $v$ satisfying  \cref{eq:continuity_equation}  with boundary conditions given by \cref{eq:boundary_conditions}.

\begin{remark}
	Such paths should not be confused with another kind of paths called \textit{mixture geodesics}. The mixture geodesic $(m_t)_{t\in[0,1]}$ from $\nu$ to $\mu$ is obtained by first choosing either $\nu$ or $\mu$ according to a Bernoulli distribution of parameter $t$ and then sampling from the chosen distribution:
	\begin{align}\label{eq:mixture_geodesic}
	m_t = (1-t)\nu + t\mu \qquad \forall t \in [0,1].
	\end{align}
	Paths of the form \cref{eq:mixture_geodesic} can be thought as the shortest paths between two distributions when distances on $\mathcal{P}_2(\X)$ are measured using the $MMD$ (\cite{Bottou:2017} Theorem 5.3). We refer to \cite{Bottou:2017} for an overview of the notion of shortest paths in probability spaces and for the differences between mixture geodesics and displacement geodesics.
	Although, we will be interested in the $MMD$ as a loss function, we will not consider the geodesics that are naturally associated to it and we will rather consider the displacement geodesics defined in \cref{eq:displacement_geodesic} for reason that will become clear in \cref{subsec:wasserstein_flow}.
\end{remark}


\subsubsection{Gradient flows on the space of probability measures}\label{sec:gradient_flows_functionals}


%Let $\F : \mathcal{P}(\X) \rightarrow \R \cup \infty$, $\rho \mapsto \F(\rho)$ a functional. %We call $\frac{\partial{\F}}{\partial{\rho}}$ if it exists, the unique (up to additive constants) function such that $\frac{d}{d\epsilon}\F(\rho+\epsilon  f)_{\epsilon=0}=\int\frac{\partial{\F}}{\partial{\rho}}(\rho) df$ for every perturbation $f$ such that, at least for $\epsilon \in [0, \epsilon_0]$, the measure $\rho +\epsilon f$ belongs to $\mathcal{P}(\X)$. The function $\frac{\partial{\F}}{\partial{\rho}}$ is called first variation of the functional $\F$ at $\rho$. 
Consider a 
\textit{Lyapunov functional} 
(or "free energy") $\F$ over the space of probability measures $\mathcal{P}(\X)$
(see \cite{Villani:2004}), 
i.e. a functional of the form:
\begin{equation}\label{eq:lyapunov}
\F(\rho)=\int U(\rho(x)) \rho(x)dx + \int V(x)\rho(x)dx + \int W(x,y)\rho(x)\rho(y)dxdy
\end{equation}
where  $U$ is the internal energy, $V$ the potential (or confinement) energy and $W$ the
interaction energy. The formal gradient flow equation associated to this functional can be written:
\begin{equation}\label{eq:continuity_equation1}
\frac{\partial \rho}{\partial t}= div( \rho \nabla \frac{\partial \F}{\partial \rho})=div( \rho \nabla (U'(\rho) + V + W * \rho))
\end{equation}
where $\nabla \frac{\partial \F}{\partial \rho}$ is the strong subdifferential of $\F$ associated with the 2-Wasserstein
metric (see \cite{ambrosio2008gradient}, Lemma 10.4.1). Indeed, for some generalized notion of gradient $\nabla_{W_2}$, and for sufficiently regular $\rho$ and $\F$, the r.h.s. of \eqref{eq:continuity_equation1} corresponds to $-\nabla_{W_2}\F(\rho)$.
The dissipation of entropy is defined as\aknote{add ref villani again}: 
\begin{align}
       \frac{d \F(\rho)}{dt} =-D(\rho) \quad \text{ with } D(\rho)= \int |\nabla \frac{\partial \F}{\partial \rho}|^2 \rho(x)dx
%&\text{ and } \xi= \nabla \frac{\partial \F}{\partial \rho} = \nabla (U'(\rho) + V + W * \rho)
\end{align}
Standard considerations from fluid mechanics tell us that the continuity equation \eqref{eq:continuity_equation1} may be interpreted as the equation ruling the evolution of the density $\rho_t$ of a family of particles initially distributed according to some $\rho_0$, and each particle follows the velocity vector field $v_t=\nabla \frac{\partial{\F}}{\partial{\rho_t}}(\rho_t)$.

\begin{remark} \label{rem:KL_Lyapunov}\aknote{define in the appendix div, laplacian...}
	A famous example of a free energy \eqref{eq:lyapunov} is the Kullback-Leibler divergence, defined for $\rho, \mu \in \mathcal{P}(\X)$ by
	$KL(\rho,\mu)=\int log(\frac{\rho(x)}{\mu(x)})\rho(x)dx$. Indeed, $KL(\rho, \mu)=\int U(\rho(x))dx + \int V(x) \rho(x)dx$ with $U(s)=s\log(s)$ the entropy function and $V(x)=-log(\mu(x))$. In this case, $\nabla \frac{\partial \F}{\partial \rho}= \nabla \log(\rho) + \nabla V=  \nabla \log(\frac{\rho}{\mu})$ and equation \eqref{eq:continuity_equation1} leads to the classical Fokker-Planck equation:
	\begin{equation}\label{eq:Fokker-Planck}
	\frac{\partial{\rho}}{\partial t}= div(\rho \nabla V )+ \Delta \rho
	\end{equation}
It is well-known (see for instance \cite{jordan1998variational}) that the distribution of the Langevin diffusion:
	\begin{equation}\label{eq:langevin_diffusion}
	dX_t= -\nabla \log \mu (X_t)dt+\sqrt{2}dB_t
	\end{equation}
	where $(B_t)_{t\ge0}$ is a $d$-dimensional Brownian motion, satisfies \eqref{eq:Fokker-Planck}.
\end{remark}


The next section describes the dynamics of the gradient flow of \cref{eq:closed_form_MMD} under the $2$-Wasserstein metric as defined in \cref{subsec:wasserstein_flow}.
%The MMD was successfully used for training generative models (\cite{mmd-gan,Binkowski:2018,Arbel:2018}) where it is used in a loss functional to learn the parameters of the generator network. This motivate the  

\subsubsection{Stochastic processes}\label{sec:ito_stochastic}

Consider the Itô process, i.e. the stochastic process:
\begin{equation}
dX_t=g(X_t)dt.
\end{equation}
Let $f$ be a twice-differentiable scalar function, Itô's formula (see \cite{ito1951stochastic}) can be written:
\begin{equation}
df(X_t)=\nabla f(X_t).g(X_t)dt
\end{equation}
Let $\rho_t$ be the distribution of the process $X_t$. We have:
\begin{align}
\E[\frac{df}{dt}(X_t)]&= \E[\nabla f(X_t).g(X_t)]\\
\Longleftrightarrow \int f(X) \frac{d \rho_t}{dt}(X)&=-\int f(X)div(g(X)\rho_t(X))
\end{align}
where the second line is obtained by integrating by parts on both sides of the equality. Finally, the distribution $\rho_t$ verifies the continuity equation: 
\begin{equation}
\frac{d\rho_t}{dt}=div(g\rho_t)
\end{equation}


\subsubsection{Additional lemmas}

\begin{lemma}\label{lem:mixture_convexity}[Mixture convexity]
	The functional $\F$ is mixture convex: for any probability distributions $\nu_1$ and $\nu_2$ and scalar $1\leq \lambda\leq 1$:
	\begin{align*}
	\F(\lambda \nu_1+(1-\lambda)\nu_2)\leq \lambda \F(\nu_1)+ (1-\lambda)\F(\nu_2)
	\end{align*}
\end{lemma}
\begin{proof}
	Let $\nu$ and $\nu'$ be two probability distributions and $0\leq \lambda\leq 1$.
	We need to show that \[\mathcal{F}(\lambda \nu + (1-\lambda)\nu') -\lambda \mathcal{F}(\nu) -(1-\lambda)\mathcal{F}(\nu')\leq 0\]
	This follows from a simple computation which shows that:
	\begin{align*}
	\mathcal{F}(\lambda \nu + (1-\lambda)\nu') -\lambda \mathcal{F}(\nu) -(1-\lambda)\mathcal{F}(\nu') = -\frac{1}{2}\lambda(1-\lambda)MMD(\nu,\nu')^2 \leq 0.
	\end{align*}
\end{proof}


\begin{lemma}\label{lem:mmd_w2}[Lipschitzness of the MMD w.r.t. the $W_1$ and $W_2$ distance]
	 Suppose that $k$ is bounded and measurable on $\X$, and that there exists $L_k$ such that $\forall x,y \in \X$, $\| k(x,.)-k(y,.) \|_{\kH}\le L_k \|x-y\|$. Then for all $\mu, \nu$ in $\mathcal{P}(\X)$:
	\begin{equation}
	MMD^2(\mu,\nu)\le  L_k^2 W_1^2(\mu,\nu) \le L_k^2 W_2^2(\mu,\nu)
	\end{equation}
\end{lemma}
\begin{proof}
Let $\mu, \nu$ in $\mathcal{P}(\X)$. By Proposition 20 in \cite{sriperumbudur2010hilbert} we have:
\begin{equation}
	MMD(\mu, \nu)	 \le \inf_{\pi \in \Pi(\mu, \nu)} \int \| k(x,.)-k(y,.) \|_{\kH} d\pi(\mu, \nu)
\end{equation}
Hence:
\begin{align}
	MMD^2(\mu, \nu)	
	 \le (\inf_{\pi \in \Pi(\mu, \nu)} \int L_k \| x-y \| d\pi(\mu, \nu))^2
 \le L_k^2 W_1^2(\mu, \nu) \le L_k^2 W_2^2(\mu,\nu)
\end{align}
\end{proof}
